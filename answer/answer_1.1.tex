\answer

\begin{xiaotis}

    \xiaoti{将下列语句形式化.}

    \begin{xiaoxiaotis}
        
        \xiaoxiaoti{2既是偶数又是素数.}

        \xiaoxiaoti{一个整数是奇数当且仅当它不能被2整除.}

        \xiaoxiaoti{除非你年满18周岁,否则你没有选举权.}

        \xiaoxiaoti{有些偶数能被3整除.}

        \xiaoxiaoti{是金子都闪光,但闪光的并不都是金子.}

    \end{xiaoxiaotis}

    \solve (1)设谓词$P(x)$表示$x$是偶数,谓词$Q(x)$表示$x$是素数,则原语句形式化后为$P(2) \land Q(2)$. 

    (2)设谓词$P(x)$表示$x$是奇数,谓词$Q(x)$表示$x$能被2整除,则原语句形式化后为$P(x) \Leftrightarrow Q(x)$. 

    (3)设命题$P$表示你年满18周岁,命题$Q$表示你有选举权,则原语句形式化后为$Q \Rightarrow P$. 

    (4)设谓词$P(x)$表示$x$是偶数,谓词$Q(x)$表示$x$能被3整除,则原语句形式化后为$\exists xP(x) \Rightarrow Q(x)$. 

    (5)设命题$P$表示一件事物是金子,命题$Q$表示一件事物会闪光,则原语句形式化后为$(P \Rightarrow Q) \land (Q \not\Rightarrow P)$. 

    \solved

    \xiaoti{命题$\exists x > 5$,$2x^2 - x + 1 > 0$的否定为\ulspace.}

    \solve $\forall x>5$,$2x^2 - x + 1 \le 0$.  

    \solved

    \spacing
    \xiaoti{判断命题$P:\exists x \in \R$,$x^2-x + \dfrac14 \ge 0$的真假,并证明之.}
    \spacing

    \solve 由于$x^2-x + \dfrac14 = (x - \dfrac12)^2 \ge 0$恒成立,所以命题$P$为真. 

    \solved

    \xiaoti{关于$x$的不等式$ax^2-2x+1>0$在$\R$上恒成立的充要条件是(\quad)}

    \fourchoices{$0<a<1$}{$0\le a<1$}{$a>1$}{$a<0$或$a>1$}

    \solve $ax^2-2x+1>0$在$\R$上恒成立当且仅当$a>0$且$\Delta = (-2)^2 - 4 \cdot a \cdot 1 = 4-4a < 0$,解之得$a > 1$. 
    故选C. 

    \solved

    \spacing
    \xiaoti{“对任意的$x\in \R$,都有$2kx^2+kx-\dfrac38 < 0$”的一个充分不必要条件是(\quad)}
    \spacing

    \fourchoices{$-3<k<0$}{$-3<k\le0$}{$-3<k<1$}{$k>-3$}

    \spacing
    \solve $2kx^2+kx-\dfrac38 < 0$在$\R$上恒成立当且仅当$k < 0$并且$\Delta = k^2 - 4 \cdot 2k \cdot \left(- \dfrac38\right) = k^2 + 3k < 0$,解之得$k < -3$. 
    故选A. 

    \solved

    \xiaoti{若命题$\exists x \in \R$,$ax^2 + 2ax + 1 \le 0$是假命题,求实数$a$的取值范围.}

    \solve 由$\exists x \in \R$,$ax^2 + 2ax + 1 \le 0$为假知$\forall x \in \R$,$ax^2 + 2ax + 1 > 0$为真. 
    而$ax^2 + 2ax + 1 > 0$在$\R$上恒成立当且仅当$a>0$且$\Delta = (2a)^2 - 4 \cdot a \cdot 1 = 4a^2 - 4a < 0$,解之得$0<a<1$. 
    故$a$的取值范围为$0<a<1$. 

    \solved

    \xiaoti{已知命题$P: \forall x \in \R$,$ax^2 - ax + 1 > 0$,命题$Q: \exists x \in \R$,$x^2+x+a=0$.
    若$P$,$Q$中有且仅有一个为真命题,求实数$a$的取值范围.}

    \solve $P$,$Q$中有且仅有一个为真命题等价于要么$P$真$Q$假,要么$P$假$Q$真. 
    故我们接下来针对$P$和$Q$的真值开始讨论. 

    \circled{1} 当$P$真$Q$假时,则$ax^2 - ax + 1 > 0$在$\R$上恒成立且$x^2+x+a \neq 0$在$\R$上恒成立. 
    $ax^2 - ax + 1 > 0$在$\R$上恒成立当且仅当$a>0$且$\Delta = (-a)^2 - 4a = a^2-4a < 0$,解之得$0<a<4$. 
    $x^2+x+a \neq 0$在$\R$上恒成立当且仅当$\Delta = 1^2 - 4a = 1-4a < 0$,解之得$a > \dfrac14$. 
    故这种情形下$a$的取值范围为$\dfrac14 < a < 4$. 

    \spacing
    事实上,我们由上面的推理得到:$P \Leftrightarrow 0<a<4$,$\lnot Q \Leftrightarrow a > \dfrac14$. 
    从而我们能够得到$\lnot P \Leftrightarrow a\le0$或$a\ge4$,$Q \Leftrightarrow a \le \dfrac14$. 
    \spacing

    \circled{2} 当$P$假$Q$真时,即$\lnot P \land Q$为真. 
    由上面的推理,我们可以得出$\lnot P \land Q \Leftrightarrow a\le0$. 

    \spacing
    综上,$a$的取值范围为$\dfrac14 < a < 4$或$a\le0$.
    \spacing

    \solved

\end{xiaotis}