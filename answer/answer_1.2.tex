\answer

\begin{xiaotis}
    
    \xiaoti{设集合$A=\{2, a^2-a+2, 1-a\}$. 
        若$4 \in A$,则$a$的值是\ulspace.
    }

    \solve 
    因为$4 \in A$,所以$a^2-a+2=4$或$1-a=4$,解得$a=2$或$-1$或$-3$. 
    (这是$4 \in A$的必要条件,需要代回检验充分性)
    当$a=2$时,$A=\{2,4,1\}$,满足题意;
    当$a=-1$时,$1-a=2$,与集合的元素互异性相悖,故应舍去;
    当$a=-3$时,$A=\{2,14,4\}$,满足题意.

    综上,答案是2或$-3$.

    \solved
    
    \xiaoti{以下关于集合的描述,正确的是(\qquad)
    }

    \onechoices
    {$\{\varnothing\}$ 是空集,因为它包含的是空集}
    {$\{\varnothing\}$ 是一个单元素集,元素是空集 $\varnothing$}
    {$\{\varnothing\}$ 和 $\varnothing$ 是相同的集合}
    {$\varnothing \in \{\varnothing\}$表明$\varnothing$是$\{\varnothing\}$的子集}

    \solve 
    选B. 
    选项 A 错误,$\{\varnothing\}$ 是一个单元素集,而不是空集. 
    选项 B 正确,$\{\varnothing\}$ 包含一个元素,即空集. 
    选项 C 错误,$\{\varnothing\}$ 和 $\varnothing$ 是不同的集合. 
    选项 D 错误,$\varnothing \in \{\varnothing\}$表明$\varnothing$是$\{\varnothing\}$的元素. 
    注意,$\varnothing \in \{\varnothing\}$与$\varnothing \subseteq \{\varnothing\}$均是正确的.

    \solved

    \xiaoti{若集合$A=\{1,2,3,m\}$,$b=\{2,3,m^2\}$,若$B \subseteq A$,则实数$m$的值为\ulspace. 
    }

    \solve 
    集合$B$中2,3这两个元素$A$中已经有了,故只需考虑元素$m^2$是$A$中的谁即可. 
    因为 $B \subseteq A$ 且 $m^2 \in B$,所以 $m^2 \in A$,故 $m^2 = 1$ 或 $m^2 = m$或 $0$. 
    (别忘了检验是否满足集合中元素互异. )
    经检验,当 $m = 1$ 时,集合 $A$ 中有相同元素,舍去;当 $m = -1$ 或 $0$ 时,集合 $A$,$B$ 均满足元素互异;所以实数 $m$ 的值为 $-1$ 或 $0$.

    \solved

    \xiaoti{已知集合$A = \{a,b,1\}$,$B=\{-1,2,a^2\}$. 
        若$A=B$,则$a^b=$(\qquad)
    }

    \spacing
    \fourchoices
    {1}
    {$\dfrac 1 2$}
    {$-1$}
    {1或$\dfrac 1 2$}
    \spacing

    \solve 
    集合$A$中有两个待定参数,故可以考虑它们是集合$B$中的谁,这样可以一次性待定出两个系数. 
    由$A=B$知要么$a=-1$,$b=2$,要么$a=2$,$b=-1$. 
    (这是$A=B$的必要条件,需要代回检验充分性)
    当$a=-1$,$b=2$时,$A=\{-1, 2, 1\}$,$B=\{-1,2,1\}$符合题意. 
    当$a=2$,$b=-1$时,$A=\{2, -1, 1\}$,$B=\{-1,2,4\}$不符合题意,应舍去. 
    综上,$a=-1$,$b=2$,$a^b=(-1)^2=1$. 

    \solved

    \xiaoti{已知集合$A=\{x\in\R \mid x^2+ax+1=0\}$,$B=\{x\in\R \mid x^2+2x-a+3=0\}$,若$A=B$,则实数$a$的取值范围是\ulspace.
    }

    \solve 
    $A$,$B$都是一元二次方程的解集,$A=B$意味着两个方程同解. 

    讨论:\circled{1} 先考虑他们都是无解的情况. 
    $A=B=\varnothing$。则$\Delta_1 = a^2-4<0$且$\Delta_2 = 2^2-4(-a+3)<0$,解之得$-2<a<2$. 

    \circled{2} 再考虑$A$,$B$不是空集的情形,此时两个一元二次方程应同解. 
    直接算出解集显然偏麻烦,所以考虑由韦达定理建立方程求$a$. 
    若$A=B\neq\varnothing$,设方程$x^2+ax+1=0$和$x^2+2x-a+3=0$的解为$x_1$,$x_2$,则首先应有$\Delta_1 = a^2-4\ge0$且$\Delta_2 = 2^2-4(-a+3)\ge0$,解之得$a\ge2$. 
    其次,由韦达定理,有$x_1+x_2=-a=-2$且$x_1+x_2=1=-a+3$,解之得$a=2$,满足$a\ge2$. 

    综上所述,实数$a$的取值范围是$-2<a\le2$. 
    
    \solved

    \xiaoti{设集合 $A = \{1,2\}$,$B = \{2,4,6\}$,则 $A \cup B =  $ (\qquad)
    }

    \fourchoices
    {$\{2\}$}
    {$\{1,2\}$}
    {$\{2,4,6\}$}
    {$\{1,2,4,6\}$}

    \solve 
    求并集,把两集合的元素合在一起即可,需注意相同元素只计一次. 
    由题意,$A \cup B = \{1,2,4,6\} $. 
    故答案选D.

    \solved

    \xiaoti{已知集合$A = \{x \mid a—2 < x <a+3 \}$,$B = \{ x \mid x^2 -5x + 4 > 0 \}$ ,若$A \cup B = \R$, 则 $a$ 的取值范围是(\qquad)
    }

    \fourchoices
    {$(-\infty, 1)$}
    {(1, 3)}
    {[1, 3]}
    {$(3, +\infty)$}

    \solve 
    $x^2 - 5x + 4 > 0 \Leftrightarrow (x-1)(x-4) > 0 \Leftrightarrow x < 1$ 或 $x > 4$,所以 $B = (-x, 1) \cup (4, +\infty)$. 
    分析连续取值集合的集合关系,常画数轴来看,其中端点能否重合需重点关注. 
    要使 $A \cup B = \R$,$a-2$ 与1,$a+3$ 与 $4$ 都不能重合,否则并集就取不到端点处的元素,所以应有$a-2 < 1$,解得 $1 < a < 3$,故 $a$ 的取值范围是 (1, 3).

    注意,对于连续取值的集合的包含关系或交并补运算,一般优先考虑画数轴分析,需要尤其注意端点取值. 

    \begin{figure}[ht]
        \hfill
        \includegraphics[width=0.3\linewidth]{./answer/image_1.2/figure7.png}
    \end{figure}

    \solved

    \xiaoti{若集合 $M = \{x \mid \sqrt{x} < 4\}$,$N = \{x \mid 3x \geq 1\}$,则$M \cap N = $(\qquad)
    }

    \spacing
    \fourchoices
    {\(\{x \mid 0 \leq x < 2\}\)}
    {\(\{x \mid \dfrac{1}{3} \leq x < 2\}\)}
    {\(\{x \mid 3 \leq x < 16\}\)}
    {\(\{x \mid \dfrac{1}{3} \leq x < 16\}\)}

    \spacing
    \solve
    $\sqrt{x} < 4 \Leftrightarrow 0 \leq x < 16$,所以$M = \{x| 0 \leq x < 16\}$,$3x \geq 1 \Leftrightarrow x \geq \dfrac{1}{3}$,所以$N = \{x| x \geq \dfrac{1}{3} \}$,所以$M \cap N = \{ x \mid \dfrac{1}{3} \leq x < 16 \}$.

    \begin{figure}[ht]
        \hfill
        \includegraphics[width=0.3\linewidth]{./answer/image_1.2/figure8.png}
    \end{figure}

    \solved

    \xiaoti{已知集合 \(A = \{ x \mid x^2 - x - 2 \leq 0 \}\),\(B = \{ x \mid 2a < x < a^2 \}\),若 \(A \cap B = \varnothing\),则实数 \(a\) 的取值范围是\ulspace.}

    \solve
    $x^2 - x - 2 \leq 0 \Leftrightarrow (x + 1)(x - 2) \leq 0 \Leftrightarrow -1 \leq x \leq 2$,所以 \(A = [-1, 2]\).    
    当 \(B = \varnothing\) 时, \(2a \geq a^2\),解得\(0 \leq a \leq 2\),此时满足 \(A \cap B = \varnothing\);
    若 \(B \neq \varnothing\) 时,首先应有 \(2a < a^2\),解得\(a < 0\) 或 \(a > 2\) \circled{1};
    其次,图形应为图1或图2所示的情形. 
    若为图1,则 \(a^2 \leq -1\),无解:
    若为图2,则 \(2a \geq 2\),解得\(a \geq 1\),结合 \circled{1} 可得 \(a > 2\):综上所述,实数 \(a\) 的取值范围是 \([0, -\infty)\)。

    \begin{figure}[ht]
        \hfill
        \includegraphics[width=0.6\linewidth]{./answer/image_1.2/figure9.png}
    \end{figure}

    \solved

\end{xiaotis}


