\section{数学规划}

规划论,又称数学规划,是运筹学的分支,研究在约束条件下通过分配资源寻求目标函数极值的数学方法,应用于经济管理、工程设计和过程控制等领域. 
其核心是建立约束条件与目标函数的数学模型,这种数学模型的一般形式为
\begin{align*}
    \text{Opt.} \quad & f(x_i, y_j, \xi_k) \\
    \text{s.t.} \quad & g_h(x_i, y_j, \xi_k) \le (=, \ge) 0 \\
    &h = 1, 2, \cdots, m
\end{align*}
其中,$x_i$是控制变量,$y_j$是已知参数,$\xi_k$是随机因素,$f$是目标函数,$g_h$是约束函数. 

按照目标函数和自变量取值范围的不同,数学规划可分为:
\begin{itemize}
    \item 线性规划:$f, g$都是线性函数;
    \item 非线性规划:$f, g$中有非线性函数;
    \item 多目标规划:$f$是向量函数;
    \item 整数规划:决策变量$x_i$为整数;
    \item 动态规划:含多阶段决策过程;
    \item 随机规划:含有随机因子.
\end{itemize}

若记
$$
S= \{ \bs{x} \mid g_h(x_i, y_j, \xi_k) \le (=, \ge) 0 \} \subseteq \R^n
$$
为约束集合,则这种数学模型可进一步简记为
$$
(f\ S)
\begin{cases}
    \min \quad & f(\bs{x}) \\
    \text{s.t.} \quad & \bs{x} \in S
\end{cases}
$$
其中$\bs{x} \in S$称模型$(f\ S)$的可行解. 

\begin{definition}{全局最优解、最优值}{}
    若对于$\bs{x}^*\in S$,满足$f(\bs{x}^*) \le f(\bs{x})$,$\forall \bs{x} \in S$,则称$\bs{x}^*$为$(f\ S)$的全局最优解(最优解),记作g.opt.,简记为opt.. 
    称此时的$f(\bs{x}^*)$为$(f\ S)$的最优值(最优目标函数值). 
\end{definition}

\begin{definition}{局部最优解}{}
    若对于$\bs{x}^*\in S$,$\exists \bs{x}^*$的某邻域$U(\bs{x}^*)$,使得$f(\bs{x}^*) \le f(\bs{x})$,$\forall \bs{x} \in S$,则称$\bs{x}^*$为$(f\ S)$的局部最优解,记作l.opt.,
\end{definition}

在上述定义中,若当$\bs{x}\neq \bs{x}^*$时有严格不等式成立,则分别称$\bs{x}^*$为$(f\ S)$的严格全局最优解和严格局部最优解. 

