\section{多元函数及其导数}

多元函数\(f: \R^n \to \R\)可看作一个\(n\)维向量\(\bs{x}\)的函数\(f(\bs{x})\). 
受篇幅所限,我们并不给出矩阵导数的定义,具体请参见《矩阵分析》课程相关内容. 
在此我们只给出梯度(一阶偏导数向量)和Hesse阵(二阶偏导数矩阵). 

\begin{definition}{梯度、Hesse阵}{}
    设函数 \( f: S \subseteq \R^n \to \R \) 在其定义域上分别一阶可微和二阶可微,则
    \begin{enumerate}
        \item 梯度 \( \nabla f(\bs{x}) = \left(\frac{\partial f}{\partial x_1}, \frac{\partial f}{\partial x_2}, \cdots , \frac{\partial f}{\partial x_n} \right)^{\top} \);
        \item Hesse阵 \( \nabla^2 f(x) = \begin{pmatrix}
            \frac{\partial^2 f}{\partial x_1^2} & \frac{\partial^2 f}{\partial x_2 \partial x_1} & \cdots & \frac{\partial^2 f}{\partial x_n \partial x_1} \\
            \frac{\partial^2 f}{\partial x_1 \partial x_2} & \frac{\partial^2 f}{\partial x_2^2} & \cdots & \frac{\partial^2 f}{\partial x_n \partial x_2} \\
            \vdots & \vdots & \ddots & \vdots \\
            \frac{\partial^2 f}{\partial x_1 \partial x_n} & \frac{\partial^2 f}{\partial x_2 \partial x_n} & \cdots & \frac{\partial^2 f}{\partial x_n^2}
        \end{pmatrix} \)
    \end{enumerate}
\end{definition}

\begin{theorem}{一些常见函数的梯度}{}
    \begin{itemize}
        \item 线性函数:\(f(\bs{x}) = \bs{c}^{\top}\bs{x} + \bs{d}\),\(\nabla f(\bs{x}) = \bs{c}\),\(\nabla^2 f(x) = \bs{0}\);
        \item 二次函数:\(f(\bs{x}) = \frac{1}{2} \bs{x}^{\top} \bs{Qx} + \bs{c}^{\top}\bs{x} + \bs{d}\),\(\nabla f(\bs{x}) = \bs{Qx} + \bs{c}\),\(\nabla^2 f(x) = \bs{Q}\),其中\(\bs{Q}\)为对称阵. 
        \item 向量值线性函数:\(\bs{f}(\bs{x}) = \bs{Ax} + \bs{d} \in \R^m\),\(\frac{\d \bs{f}}{\d \bs{x}} = \bs{A}^{\top}\). 
    \end{itemize}
\end{theorem}

\eg
求下列函数的梯度和Hesse矩阵. 
\begin{enumerate}
    \item \( f_1(\bs{x}) = x^1_2 - x_1 + x^2_2 + x_1 x_2 + 9 \) ;
    \item \( f_2(\bs{x}) = 2x^1_2 + x^2_2 + 5x^2_3 + 3x_1 x_2 - 6x_1 x_3 + 4x_1 x_2 x_3 + 17 \);
    \item \( f_3(\bs{x}) = 10 - (x_2 - x_1^2)^2\). 
\end{enumerate}

\begin{solution}

\end{solution}

接下来依旧不加证明地介绍多元函数的Taylor展开式及中值定理. 
此处不加证明是因为此内容可以被读者在任何一本《数学分析》教材中找到. 

\begin{theorem}{Taylor展开式}{}
    设\(f: \R^n \to \R\)在\(\bs{x}^*\)的某邻域内分别一阶、二阶可微,则分别有:
    \begin{itemize}
        \item 一阶Taylor展开式:\(f(\bs{x}) = f(\bs{x}^*) + \nabla f^{\top}(\bs{x}^*) (\bs{x} - \bs{x}^*) + o(\Vert \bs{x} - \bs{x}^* \Vert)\);
        \item 二阶Taylor展开式:\(f(\bs{x}) = f(\bs{x}^*) + \nabla f^{\top}(\bs{x}^*) (\bs{x} - \bs{x}^*)  + \frac{1}{2} (\bs{x} - \bs{x}^*)^{\top} \nabla^2 f(\bs{x}^*) (\bs{x} - \bs{x}^*) + o(\Vert \bs{x} - \bs{x}^* \Vert^2)\). 
    \end{itemize}
\end{theorem}

\begin{theorem}{中值定理}{}
    设\(f: \R^n \to \R\)在\(\bs{x}^*\)的某邻域\(U(\bs{x}^*)\)内二阶可导,则\(\forall x \in U(\bs{x}^*)\),都有:
    \begin{itemize}
        \item Lagrange中值定理:\(\exists \lambda \in (0,1)\),记\(\bs{x}_{\lambda} = \bs{x^*} + \lambda (\bs{x} - \bs{x}^*)\),则有
        \[
            f(\bs{x}) = f(\bs{x}^*) + \nabla f^{\top} (\bs{x}_{\lambda}) (\bs{x} - \bs{x}^*)
        \]
        \item Taylor中值定理:\(\exists \mu \in (0,1)\),记\(\bs{x}_{\mu} = \bs{x^*} + \mu (\bs{x} - \bs{x}^*)\),则有
        \[
          f(\bs{x}) = f(\bs{x}^*) + \nabla f^{\top}(\bs{x}^*) (\bs{x} - \bs{x}^*)  + \frac{1}{2} (\bs{x} - \bs{x}^*)^{\top} \nabla^2 f(\bs{x}_{\mu}) (\bs{x} - \bs{x}^*)
        \]
    \end{itemize}
\end{theorem}

\begin{definition}{方向导数}{directional_derivative}
    设\(S \subseteq \R^n\)为非空凸集,函数\(f: S \to \R\),\(\bs{x}^* \in S\),方向\(\bs{d} \in \R^n\),若\(\exists \lambda > 0\),使得\(\bs{x}^* + \lambda \bs{d} \in S\)(通常令\(\lambda\)充分小,此时称\(\bs{d}\)为{\bf 可行方向}),且此时极限
    \[
        \lim_{\lambda \to 0^+} \frac{f(\bs{x}^* + \lambda \bs{d}) - f(\bs{x}^*)}{\lambda}
    \]
    存在,则称\(f(\bs{x})\)在点\(\bs{x}^*\)沿方向\(\bs{d}\)的{\bf 方向导数}存在,记作\(f'(\bs{x}^*; \bs{d})\). 
\end{definition}

\begin{theorem}{方向导数和梯度的关系}{}
    若\(f(\bs{x})\)在\(\bs{x}^*\)处可微,则
    \[
        f'(\bs{x}^*; \bs{d}) = \nabla f^{\top} (\bs{x}^*) \bs{d}
    \]
\end{theorem}

\begin{proof}
    应用\(f(\bs{x})\)的一阶展开式,则
    \begin{align*}
        \lim_{\lambda \to 0^+} \frac{f(\bs{x}^* + \lambda \bs{d}) - f(\bs{x}^*)}{\lambda}
        & = \lim_{\lambda \to 0^+}\frac{f(\bs{x}^*) + \nabla f^{\top}(\bs{x}^*) (\lambda \bs{d}) + o(\lambda \Vert \bs{d} \Vert)}{\lambda} \\
        & = \nabla f^{\top} (\bs{x}^*) \bs{d} + \lim_{\lambda \to 0^+} \frac{o(\lambda \Vert \bs{d} \Vert)}{\lambda} \\
        & = \nabla f^{\top} (\bs{x}^*) \bs{d} + \lim_{\lambda \to 0^+} \frac{o(\lambda \Vert \bs{d} \Vert)}{\lambda \Vert \bs{d} \Vert} \cdot \Vert \bs{d} \Vert \\
        & = \nabla f^{\top} (\bs{x}^*) \bs{d}
    \end{align*}
\end{proof}
