\section{多元函数及其导数}

多元函数$f: \R^n \to \R$可看作一个$n$维向量$\bs{x}$的函数$f(\bs{x})$. 
受篇幅所限,我们并不给出矩阵导数的定义,具体请参见《矩阵分析》课程相关内容. 
在此我们只不加证明地给出一些常用函数的矩阵导数(即其梯度). 

\begin{theorem}
    {\bf(一些常见函数的梯度)}

    \begin{itemize}
        \item 线性函数:$f(\bs{x}) = \bs{c}^{\top}\bs{x} + \bs{d}$,$\nabla f(\bs{x}) = \bs{c}$,$\nabla^2 f(x) = \bs{0}$;
        \item 二次函数:$f(\bs{x}) = \frac{1}{2} \bs{x}^{\top} \bs{Qx} + \bs{c}^{\top}\bs{x} + \bs{d}$,$\nabla f(\bs{x}) = \bs{Qx} + \bs{c}$,$\nabla^2 f(x) = \bs{Q}$,其中$\bs{Q}$为对称阵. 
        \item 向量值线性函数:$\bs{f}(\bs{x}) = \bs{Ax} + \bs{d} \in \R^m$,$\frac{\d \bs{f}}{\d \bs{x}} = \bs{A}^{\top}$. 
    \end{itemize}
\end{theorem}

接下来依旧不加证明地介绍多元函数的Taylor展开式及中值定理. 
此处不加证明是因为此内容可以被读者在任何一本《数学分析》教材中找到. 

\begin{theorem}
    {\bf(Taylor展开式)}

    设$f: \R^n \to \R$在$\bs{x}^*$的某邻域内分别一阶、二阶可微,则分别有:
    \begin{itemize}
        \item 一阶Taylor展开式:$f(\bs{x}) = f(\bs{x}^*) + \nabla f^{\top}(\bs{x}^*) (\bs{x} - \bs{x}^*) + o(\Vert \bs{x} - \bs{x}^* \Vert)$;
        \item 二阶Taylor展开式:$f(\bs{x}) = f(\bs{x}^*) + \nabla f^{\top}(\bs{x}^*) (\bs{x} - \bs{x}^*)  + \frac{1}{2} (\bs{x} - \bs{x}^*)^{\top} \nabla^2 f(\bs{x}^*) (\bs{x} - \bs{x}^*) + o(\Vert \bs{x} - \bs{x}^* \Vert^2)$. 
    \end{itemize}
\end{theorem}

\begin{theorem}
    {\bf(中值定理)}

    设$f: \R^n \to \R$在$\bs{x}^*$的某邻域$U(\bs{x}^*)$内二阶可导,则$\forall x \in U(\bs{x}^*)$,都有:
    \begin{itemize}
        \item Lagrange中值定理:$\exists \lambda \in (0,1)$,记$\bs{x}_{\lambda} = \bs{x^*} + \lambda (\bs{x} - \bs{x}^*)$,则有$f(\bs{x}) = f(\bs{x}^*) + \nabla f^{\top} (\bs{x}_{\lambda}) (\bs{x} - \bs{x}^*)$;
        \item Taylor中值定理:$\exists \mu \in (0,1)$,记$\bs{x}_{\mu} = \bs{x^*} + \mu (\bs{x} - \bs{x}^*)$,则有$f(\bs{x}) = f(\bs{x}^*) + \nabla f^{\top}(\bs{x}^*) (\bs{x} - \bs{x}^*)  + \frac{1}{2} (\bs{x} - \bs{x}^*)^{\top} \nabla^2 f(\bs{x}_{\mu}) (\bs{x} - \bs{x}^*)$. 
    \end{itemize}
\end{theorem}

\begin{definition}
    {\bf(方向导数)}
    \label{directional_derivative}

    设$S \subseteq \R^n$为非空凸集,函数$f: S \to \R$,$\bs{x}^* \in S$,方向$\bs{d} \in \R^n$,若$\exists \lambda > 0$,使得$\bs{x}^* + \lambda \bs{d} \in S$(通常令$\lambda$充分小,此时称$\bs{d}$为{\bf 可行方向}),且此时极限
    $$
        \lim_{\lambda \to 0^+} \frac{f(\bs{x}^* + \lambda \bs{d}) - f(\bs{x}^*)}{\lambda}
    $$
    存在,则称$f(\bs{x})$在点$\bs{x}^*$沿方向$\bs{d}$的{\bf 方向导数}存在,记作$f'(\bs{x}^*; \bs{d})$. 
\end{definition}

\begin{theorem}
    {\bf(方向导数和梯度的关系)}

    若$f(\bs{x})$在$\bs{x}^*$处可微,则
    $$
        f'(\bs{x}^*; \bs{d}) = \nabla f^{\top} (\bs{x}^*) \bs{d}
    $$
\end{theorem}

\prove
应用$f(\bs{x})$的一阶展开式,则
\begin{align*}
    \lim_{\lambda \to 0^+} \frac{f(\bs{x}^* + \lambda \bs{d}) - f(\bs{x}^*)}{\lambda}
    & = \lim_{\lambda \to 0^+}\frac{f(\bs{x}^*) + \nabla f^{\top}(\bs{x}^*) (\lambda \bs{d}) + o(\lambda \Vert \bs{d} \Vert)}{\lambda} \\
    & = \nabla f^{\top} (\bs{x}^*) \bs{d} + \lim_{\lambda \to 0^+} \frac{o(\lambda \Vert \bs{d} \Vert)}{\lambda} \\
    & = \nabla f^{\top} (\bs{x}^*) \bs{d} + \lim_{\lambda \to 0^+} \frac{o(\lambda \Vert \bs{d} \Vert)}{\lambda \Vert \bs{d} \Vert} \cdot \Vert \bs{d} \Vert \\
    & = \nabla f^{\top} (\bs{x}^*) \bs{d}
\end{align*}
\proved
