\section{多面体、极点、极方向}

\begin{definition}{半空间}{}
    设超平面$H: \bs{\alpha}^{\top} \bs{x} = \beta \subseteq \R^n$,则这个超平面把$\R^n$分成两部分$\{\bs{x} \in \R^n \mid \bs{\alpha}^{\top} \bs{x} \ge \beta \}$和$\{\bs{x} \in \R^n \mid \bs{\alpha}^{\top} \bs{x} \le \beta \}$,这样的两部分称为{\bf 闭半空间},若有严格不等式成立,则称其为{\bf 开半空间}. 
\end{definition}

\begin{definition}{多面体}{}
    有限个闭半空间的交称为{\bf 多面体}. 
\end{definition}

由定义看出,多面体实际上由一系列线性不等式组定义,故一般可以将其写作$\{ \bs{x} \in \R^n \mid \bs{Ax} \le \bs{b} \}$,其中$\bs{A} \in \R^{m \times n}$,$\bs{b} \in \R^m$. 
其中我们选取一种特殊的多面体$\{ \bs{x} \in \R^n \mid \bs{Ax} = \bs{b} , \bs{x} \ge \bs{0} \}$,称其为{\bf 标准型多面体}. 

\begin{definition}{极点}{}
    设$S \subseteq \R^n$,$\bs{x} \in S$,若不存在$\bs{x}^{(1)}, \bs{x}^{(2)} \in S$,$\bs{x}^{(1)} \neq \bs{x}^{(2)}$,使得$\bs{x} = \lambda \bs{x}^{(1)} + (1 - \lambda) \bs{x}^{(2)}$,则称$x$是$S$的{\bf 极点}(或{\bf 顶点}). 
\end{definition}

\begin{definition}{极方向}{}
    
    设$S \subseteq \R^n$,$\bs{d} \in \R^n$,若$\bs{d}$是$S$的可行方向(见{\bf 方向导数}的定义),且不能被表示为$S$的两个不同可行方向的非负组合,则称$\bs{d}$是$S$的{\bf 极方向}. 
\end{definition}

\begin{theorem}{极点特征}{}
    设多面体$S = \{ \bs{x} \in \R^n \mid \bs{Ax} = \bs{b}, \bs{x} \ge \bs{0} \}$ ,其中  $\bs{A} \in \R^{m \times n}$行满秩,则$\bs{x}$是多面体$S$的极点的充要条件是存在分解$\bs{A} = (\bs{B}, \bs{N})$,使得
    \begin{romlist}
        \item $B$为$m$阶可逆阵;
        \item $\bs{x} = (\bs{x_B}^{\top}, \bs{x_N}^{\top})^{\top}$(分解方式与$\bs{A}$相同),其中$\bs{x_B} = \bs{B}^{-1} \bs{b} \ge \bs{0}$,$\bs{x_N} = \bs{0}$. 
    \end{romlist}
\end{theorem}

\prove

\proved

\eg
证明:多面体$S = \{ \bs{x} \in \R^n \mid \bs{Ax} = \bs{b}, \bs{x} \ge \bs{0} \}$中必存在有限多的极点. 

\prove

\proved

\eg 
设多面体$S = \{ \bs{x} \in \R^n \mid \bs{Ax} = \bs{b}, \bs{x} \ge \bs{0} \}$,在以下条件下,求该多面体的极点:

(1)$A = \begin{pmatrix}
    3 & 1 & 2 & 0 \\
    0 & 4 & 5 & 1
\end{pmatrix}
, b = \begin{pmatrix}
    3 \\
    6
\end{pmatrix}
$;

(2)$A = \begin{pmatrix}
    3 & 2 & 1 & 0 & 0 \\
    2 & 1 & 0 & 1 & 0 \\
    0 & 3 & 0 & 0 & 1
\end{pmatrix}
, b = \begin{pmatrix}
    65 \\
    40 \\
    75
\end{pmatrix}
$. 

\solve

\solved

\begin{theorem}{极方向特征}{}
    设多面体$S = \{ \bs{x} \in \R^n \mid \bs{Ax} = \bs{b}, \bs{x} \ge \bs{0} \}$ ,其中  $\bs{A} \in \R^{m \times n}$行满秩,则
    \begin{enumerate}
        \item $\bs{d}$是$S$的方向的充要条件是$\bs{Ad} = \bs{0}$且$\bs{d} \ge 0$;
        \item $\bs{d}$是$S$的极方向的充要条件是存在分解$\bs{A} = (\bs{B}, \bs{N})$,使得
        \begin{indromlist}
            \item $\bs{B}$为$m$阶可逆阵;
            \item $\bs{d} = (\bs{d_B}^{\top}, \bs{d_N}^{\top})^{\top}$(分解方式与$\bs{A}$相同),其中$\bs{d_B} = - \bs{B}^{-1} \bs{a}_j \ge \bs{0}$,$\bs{d_N} = \bs{e}_j$. 
            这里$\bs{e}_j$是第$j$个元素为1,其余元素为0的单位向量. 
            $j$是使得$\bs{N}$中的列向量$\bs{a}_j$满足$\bs{B}^{-1} \bs{a}_j \le \bs{0}$的下标.
        \end{indromlist}
    \end{enumerate}  
\end{theorem}

\prove

\proved

\eg
证明:多面体$S = \{ \bs{x} \in \R^n \mid \bs{Ax} = \bs{b}, \bs{x} \ge \bs{0} \}$中必有极点,且个数不超过$(n-m)C_n^m$. 

\prove

\proved

\eg 
设多面体$S = \{ \bs{x} \in \R^n \mid \bs{Ax} = \bs{b}, \bs{x} \ge \bs{0} \}$,其中$A = 
\begin{pmatrix}
    3 & 1 & 2 & -6 & 0 \\
    0 & 4 & 5 & -5 & 1
\end{pmatrix}, 
b = \begin{pmatrix}
    3 \\ 6
\end{pmatrix}$,求该多面体的极方向. 

\solve

\solved

\begin{theorem}{表示定理,Minkowski-Weyl}{Minkowski-Weyl}
    设多面体$S = \{ \bs{x} \in \R^n \mid \bs{Ax} = \bs{b}, \bs{x} \ge \bs{0} \}$ ,其中  $\bs{A} \in \R^{m \times n}$行满秩,$\bs{x}^{(1)}, \bs{x}^{(2)}, \cdots, \bs{x}^{(k)}$为所有极点,$\bs{d}^{(1)}, \bs{d}^{(2)}, \cdots, \bs{d}^{(l)}$为所有极方向,则$\forall \bs{x} \in S$,存在序列$\{\lambda_i\}_{i=1}^k, \{\mu_j\}_{j=1}^l$,其中:
    \begin{enumerate}
        \item $\lambda_i >0$,$i = 1, 2, \cdots, k$,且$\displaystyle \sum_{i=1}^k \lambda_i = 1$;
        \item $\mu_j \ge 0$,$j = 1, 2, \cdots, l$,使得
        \begin{align*}
            \bs{x} = & \lambda_1 \bs{x}^{(1)} + \lambda_2 \bs{x}^{(2)} + \cdots + \lambda_k \bs{x}^{(k)} & \text{(极点的凸组合)} \\
            + & \mu_1 \bs{d}^{(1)} + \mu_2 \bs{d}^{(2)} + \cdots + \mu_l \bs{d}^{(l)} & \text{(极方向的半正组合)}
        \end{align*}
        特别地,如果上述多面体$S$是有界的,则有$\mu_j = 0$. 
    \end{enumerate}
\end{theorem}

\prove

\proved
