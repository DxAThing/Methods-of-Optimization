\section{线性规划的对偶}

\subsection{对偶问题}

我们用下面的例子引出本节的讨论. 

设某工厂有A、B、C三种类型的设备,生产甲、乙两种设备. 
每件产品在生产中需要占用的设备机时数、每件产品可以获得的利润以及三种设备可利用的机时数如下表所示:

\begin{longtblr}[
    label = none,
    entry = none,
]{
    cells = {c},
    vline{2} = {-}{0.05em},
    hline{1,6} = {-}{0.08em},
    hline{2} = {-}{0.05em},
}
 & 甲产品 & 乙产品 & 设备能力(h) \\
设备A & 3 & 2 & 65 \\
设备B & 2 & 1 & 40 \\
设备C & 0 & 3 & 75 \\
利润(元/件) & 1500 & 2500 & / 
\end{longtblr}

\noindent
试问:
\begin{enumerate}
    \item 工厂应如何安排生产可获得最大的总利润?
    \item 若工厂的设备都用于外协加工,工厂收取加工费. 
    试问:设备A、B、C每工时各如何收费才最有竞争力?
\end{enumerate}

对于第(1)个问题,设工厂生产 \( x_1 \) 个甲产品, \( x_2 \) 个乙产品,则可列线性规划问题如下:
\begin{align*}
    \max & \quad z = 1500x_1 + 2500 x_2 \\
    \text{s.t.} & \quad 3 x_1 + 2 x_2 \le 65 \\
    & \quad 2 x_1 + x_2 \le 40 \\
    & \quad 3 x_2 \le 75 \\
    & \quad x_1, x_2 \ge 0
\end{align*}

对于第(2)个问题,设设备A、B、C每工时各收费 \( y_1, y_2, y_3 \) 元,则可列线性规划问题如下:
\begin{align*}
    \min & \quad f = 65y_1 + 40y_2 + 75y_3 \\
    \text{s.t.} & \quad 3 y_1 + 2 y_2 \ge 1500 \\
    & \quad 2 y_1 + y_2 + 3y_3 \ge 2500 \\
    & \quad y_1, y_2, y_3 \ge 0
\end{align*}

% \begin{definition}{对偶规划}{}
%     称具有以下对称形式的两个线性规划问题互为对偶规划:

%     \noindent % 取消首行缩进,确保宽度计算准确
%     \begin{minipage}[t]{0.48\linewidth}
%         \begin{align*}
%             \max &\quad z = \bs{c}^{\top}\bs{x} \\
%             \text{s.t.} & \quad \bs{Ax} \le \bs{b} \\
%             & \quad \bs{x} \ge \bs{0}
%         \end{align*}
%     \end{minipage}
%     \hfill
%     \begin{minipage}[t]{0.48\linewidth}
%         \begin{align*}
%             \min &\quad f = \bs{b}^{\top} \bs{y} \\
%             \text{s.t.} & \quad \bs{A}^{\top}\bs{y} \ge \bs{c} \\
%             & \quad \bs{y} \ge \bs{0}
%         \end{align*}
%     \end{minipage}
% \end{definition}

\begin{definition}{对偶规划}{}
    称具有以下对称形式的两个线性规划问题互为对偶规划:

    % \vspace{0.5em} % 手动微调定义文字与公式的间距
    \noindent
    \begin{minipage}[t]{0.48\linewidth}
        \centering
        \( % 使用内联模式配合 aligned 消除 align* 的自带大间距
        \begin{aligned}[t]
            \text{(LP)}\max &\quad z = \bs{c}^{\top}\bs{x} \\
            \text{s.t.} & \quad \bs{Ax} \le \bs{b} \\
            & \quad \bs{x} \ge \bs{0}
        \end{aligned}
        \)
    \end{minipage}
    \hfill
    \begin{minipage}[t]{0.48\linewidth}
        \centering
        \(
        \begin{aligned}[t]
            \text{(DP)}\min &\quad f = \bs{b}^{\top} \bs{y} \\
            \text{s.t.} & \quad \bs{A}^{\top}\bs{y} \ge \bs{c} \\
            & \quad \bs{y} \ge \bs{0}
        \end{aligned}
        \)
    \end{minipage}
\end{definition}

一对对称形式的对偶规划之间具有下面的对应关系
\footnote{ \( \bs{c} \) 换 \( \bs{b} \) ,\( \bs{x} \) 换 \( \bs{y} \),约束矩阵变转置. }
:
\begin{itemize}
    \item 若一个模型为目标求“极大”,约束为“小于等于”的不等式,则它的对偶模型为目标求“极小”,约束是“大于等于”的不等式. 
    即“\( \max, \le \)”和“\( \min, \ge \)”相对应;
    \item 从约束系数矩阵看:一个模型中为 \( \bs{A} \) ,则另一个模型中为 \( \bs{A}^{\top} \) . 
    一个模型是 \( m \) 个约束、\( n \) 个变量,则它的对偶模型为 \( n \) 个约束、 \( m \) 个变量;
    \item 从数据 \( \bs{b}, \bs{c} \) 的位置看:在两个规划模型中, \( \bs{b} \) 和 \( \bs{c} \) 的位置对换;
    \item 两个规划模型中的变量皆非负. 
\end{itemize}

对于不满足上述定义形式的线性规划问题,其对偶规划可按如下规则直接给出:
\begin{itemize}
    \item 将模型统一为“\( \max, \le \)”或“\( \min, \ge \)”的形式;
    \item 若原规划的某个约束条件为等式约束,则在对偶规划中与此约束对应的变量取值没有非负限制;
    \item 若原规划的某个变量的值没有非负限制,则在对偶问题中与此变量对应的约束为等式. 
\end{itemize}

\eg
写出下面线性规划的对偶规划模型. 
\begin{align*}
    \max & \quad z = x_1 - x_2 + 5x_3 - 7x_4 \\
    \text{s.t.} & \quad x_1 + 3x_2 - 2x_3 + x_4 = 25 \\
    & \quad 2x_1 + 7x_2 + 2x_4 \ge -60 \\
    & \quad 2x_1 + 2x_2 - 4x_3 \le 30 \\
    & \quad x_1, x_2 \ge 0, -5 \le x_4 \le 10
\end{align*}

\begin{solution}
    
\end{solution}

\subsection{对偶定理}

\begin{theorem}{弱对偶定理}{}
    若 \( \bs{x}, \bs{y} \) 分别为(LP)和(DP)的可行解,那么 \( \bs{c}^{\top} \bs{x} \le \bs{b}^{\top}\bs{y} \)
\end{theorem}

\begin{proof}
    
\end{proof}

\begin{corollary*}{}{}
    设(LP)有可行解,那么若(LP)无有界最优解,则(DP)无可行解. 
\end{corollary*}

\begin{proof}
    
\end{proof}

\begin{theorem}{最优性准则}{}
    若 \( \bs{x}, \bs{y} \) 分别为(LP)和(DP)的可行解,且 \( \bs{c}^{\top} \bs{x} = \bs{b}^{\top}\bs{y} \) ,那么 \( \bs{x}, \bs{y} \) 分别为(LP)和(DP)的最优解. 
\end{theorem}

\begin{proof}
    
\end{proof}

\begin{theorem}{主对偶定理}{}
    若(LP)有最优解,则(DP)也有最优解. 
    反之也成立,且最优值相等. 
\end{theorem}

\begin{proof}
    
\end{proof}


