\section{线性规划模型}

\subsection{线性规划的定义}

\begin{definition}{线性规划的一般形式}{}

    \begin{align*}
        \max (\min) & \quad z = c_1x_1 + c_2 x_2 + \cdots + c_n x_n \\
        \text{s.t.} & \quad a_{11} x_1 + a_{12} x_2 + \cdots + a_{1n} x_n \le (=, \ge) b_1 \\
        & \quad a_{21} x_1 + a_{22} x_2 + \cdots + a_{2n} x_n \le (=, \ge) b_2 \\
        & \quad \vdots \\
        & \quad a_{m1} x_1 + a_{m2} x_2 + \cdots + a_{mn} x_n \le (=, \ge) b_m \\
        & \quad x_1, x_2, \cdots, x_n \in \R
    \end{align*}
\end{definition}

我们为了方便解决问题,常把上述一般形式转化为标准形式:

\begin{definition}{线性规划的标准形式}{}

    \begin{align*}
        \max & \quad z = c_1x_1 + c_2 x_2 + \cdots + c_n x_n \\
        \text{s.t.} & \quad a_{11} x_1 + a_{12} x_2 + \cdots + a_{1n} x_n = b_1 \\
        & \quad a_{21} x_1 + a_{22} x_2 + \cdots + a_{2n} x_n = b_2 \\
        & \quad \vdots \\
        & \quad a_{m1} x_1 + a_{m2} x_2 + \cdots + a_{mn} x_n = b_m \\
        & \quad x_1, x_2, \cdots, x_n \ge 0
    \end{align*}
    其矩阵形式为:
    \begin{align*}
        \max & \quad z = \bs{c}^{\top} \bs{x} \\
        \text{s.t.} & \quad \bs{Ax} = \bs{b} \\
        & \quad \bs{x} \ge \bs{0}
    \end{align*}
    其中,$\bs{c}, \bs{x} \in \R^n$,$\bs{b} \in \R^m$,$\bs{A} \in \R^{m \times n}$. 
\end{definition}

不难发现,标准形式有区别于一般形式的以下四个特点:
\begin{itemize}
    \item 目标最大化;
    \item 约束为等式;
    \item 决策变量均非负;
    \item 右端项非负. 
\end{itemize}

接下来我们将会讨论如何将一般形式一步步转化为标准形式. 

\subsection{将一般形式转化为标准形式}

\subsubsection{极小目标极大化}

在标准形式中,要求优化目标为极大化目标函数值. 
若目标函数为
$$
    \min f = c_1x_1 + c_2 x_2 + \cdots + c_n x_n
$$
则令$z = -f$,从而上述极小化问题与下面的极大化问题
$$
    \max z = - c_1x_1 - c_2 x_2 - \cdots - c_n x_n
$$
有相同的最优解,但它们的最优解的目标函数值相差一个符号,即
$$
    \min f = - \max z
$$

\subsubsection{约束条件不为等式}

在标准形式中,要求每约束条件均为等式. 
若约束条件为小于等于号,即设第$i$个约束条件为
$$
    a_{i1} x_1 + a_{i2} x_2 + \cdots + a_{in} x_n \le b_i
$$
则可以引入一个松弛变量$s$,使得
$$
    s = b_i - (a_{i1} x_1 + a_{i2} x_2 + \cdots + a_{in} x_n)
$$
从而$s \ge 0$,也具有非负约束,这时新的约束条件成为
$$
    a_{i1} x_1 + a_{i2} x_2 + \cdots + a_{in} x_n + s = b_i
$$

若约束条件为大于等于号,即设第$i$个约束条件为
$$
    a_{i1} x_1 + a_{i2} x_2 + \cdots + a_{in} x_n \ge b_i
$$
相似地,也可以引入一个松弛变量$s$,使得
$$
    s = (a_{i1} x_1 + a_{i2} x_2 + \cdots + a_{in} x_n) - b_i
$$
从而$s \ge 0$,也具有非负约束,这时新的约束条件成为
$$
    a_{i1} x_1 + a_{i2} x_2 + \cdots + a_{in} x_n - s = b_i
$$

\subsubsection{变量无符号限制}

在标准形式中,要求每一个变量均有非负约束. 
当某一个变量$x_j$没有非负约束时,可以令
$$
    x_j = x_j' - x_j''
$$
其中$x_j', x_j'' \ge 0$,即用两个非负数之差表示一个无符号限制的变量. 

\subsubsection{右端项有负值}

在标准形式中,要求每一个约束条件右端均为正值. 
当某一个右端项系数为负时,假设是第$i$个,即
$$
    a_{i1} x_1 + a_{i2} x_2 + \cdots + a_{in} x_n = b_i \le 0
$$
则把该等式两端同时乘以$-1$,得到
$$
    - a_{i1} x_1 - a_{i2} x_2 - \cdots - a_{in} x_n = - b_i \ge 0
$$


