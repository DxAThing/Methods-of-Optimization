\section{凸集、凸函数和凸规划}

\subsection{凸集}

\subsubsection{凸集、凸组合}

\begin{definition}
    {\bf(凸集)}

    设$S \subseteq \R^n$,若$\forall \bs{x}^{(1)}, \bs{x}^{(2)}\in S$,$\lambda \in [0,1]$,必有
    $$
        \lambda \bs{x}^{(1)} + (1 - \lambda) \bs{x}^{(2)} \in S
    $$
    则称$S$为{\bf 凸集}. 
\end{definition}

规定单点集$\{\bs{x}\}$和空集$\varnothing$都是空集. 
需要注意的是,方程$\bs{x} = \lambda \bs{x}^{(1)} + (1 - \lambda) \bs{x}^{(2)} = \bs{x}^{(2)} + \lambda (\bs{x}^{(1)} - \bs{x}^{(2)})$,$\lambda \in [0,1]$表示连接$\bs{x}^{(1)}$和$\bs{x}^{(2)}$的线段. 

\eg
设$\bs{A} \in \R^{m \times n}$,$\bs{b} \in \R^n$,证明以下命题成立:

(1)集合$S = \{ \bs{x} \mid \bs{Ax} = \bs{b} \}$是凸集;

(2)集合$S = \{ \bs{x} \mid \bs{Ax} \le \bs{b} \}$是凸集. 

\prove
(1)任取$\bs{x}^{(1)}, \bs{x}^{(2)} \in S$,$\lambda \in [0,1]$,则都有
\begin{align*}
    &\bs{A}(\lambda \bs{x}^{(1)} + (1 - \lambda) \bs{x}^{(2)}) \\
    =& \lambda \bs{A} \bs{x}^{(1)} + (1 - \lambda) \bs{A} \bs{x}^{(2)} \\
    =& \lambda \bs{b} + (1 - \lambda) \bs{b} \\
    =& \bs{b}
\end{align*}
从而$\lambda \bs{x}^{(1)} + (1 - \lambda) \bs{x}^{(2)} \in S$,故$S$是凸集. 

(2)任取$\bs{x}^{(1)}, \bs{x}^{(2)} \in S$,$\lambda \in [0,1]$,则都有
\begin{align*}
    &\bs{A}(\lambda \bs{x}^{(1)} + (1 - \lambda) \bs{x}^{(2)}) \\
    =& \lambda \bs{A} \bs{x}^{(1)} + (1 - \lambda) \bs{A} \bs{x}^{(2)} \\
    \le& \lambda \bs{b} + (1 - \lambda) \bs{b} \\
    =& \bs{b}
\end{align*}
从而$\lambda \bs{x}^{(1)} + (1 - \lambda) \bs{x}^{(2)} \in S$,故$S$是凸集. 
\proved

\begin{theorem}
    {\bf(凸集的性质)}

    (1)凸集的交是凸集;

    (2)凸集的内点集
    \footnote{
        设$S \subseteq \R^n$,$\bs{x} \in S$,如果存在一个开球$B(\bs{x}, \varepsilon) = \{ \bs{y} \in \R^n \mid \Vert \bs{y} - \bs{x} \Vert < \varepsilon \}$,使得$B(\bs{x}, \varepsilon) \subseteq S$,则称$\bs{x}$是$S$的{\bf 内点},$S$的内点的全体称为$S$的{\bf 内点集},记作$\text{int}(S)$. 
        所以$\bs{x} \in \text{int}(S) \Leftrightarrow \exists \varepsilon > 0, (\Vert \bs{y} - \bs{x} \Vert < \varepsilon \Rightarrow \bs{y} \in S)$. 
    }
    是凸集;

    (3)凸集的闭包
    \footnote{
        设$S \subseteq \R^n$,$\bs{x} \in \R^n$,如果$\forall \varepsilon > 0$都有$B(\bs{x}, \varepsilon) \cap S \neq \varnothing$,则称$\bs{x}$是$S$的{\bf 聚点},$S$的聚点的全体称为$S$的{\bf 闭包},记作$\overline{S}$. 
    }
    是凸集;

    (4)凸集边界上任意点存在支撑超平面
    \footnote{
        设$\bs{\alpha} \in \R^n \textbackslash \{ \bs{0} \}$,$\beta \in \R$,称集合$H = \{ \bs{x} \in \R^n \mid \bs{\alpha}^{\top} \bs{x} = \beta\}$是{\bf 超平面},简记作$H: \bs{\alpha}^{\top} \bs{x} = \beta$. 
        设$S \subseteq \R^n$,$\bs{x}_0 \in S$,若超平面$H: \bs{\alpha}^{\top} \bs{x} = \beta$满足:

        (i)$\exists \bs{x}^{(0)} \in S$,使得$\bs{x}^{(0)} \in H$;

        (ii)$\forall \bs{x} \in S$,$\bs{\alpha}^{\top} \bs{x} \le \beta$

        则称$H$是$S$的{\bf 支撑超平面}. 
    };

    (5)两个互相不交的凸集之间存在分离超平面
    \footnote{
        设$S_1, S_2 \subseteq \R^n$,且$S_1 \cap S_2 = \varnothing$. 
        若存在超平面$H: \bs{\alpha}^{\top} \bs{x} = \beta$,使得$\forall \bs{x}^{(1)} \in S_1$,$\bs{\alpha}^{\top} \bs{x}^{(1)} \ge \beta$,且$\forall \bs{x}^{(2)} \in S_1$,$\bs{\alpha}^{\top} \bs{x}^{(2)} \le \beta$,则称$H$是$S_1, S_2$的{\bf 分离超平面}. 
    }. 
\end{theorem}

\prove
(1)设$S_1, S_2$都是凸集,即证明$S_1 \cap S_2$是凸集,也即证明$\forall \bs{x}^{(1)}, \bs{x}^{(2)} \in (S_1 \cap S_2)$,$\forall \lambda \in [0,1]$,都有$\lambda \bs{x}^{(1)} + (1 - \lambda) \bs{x}^{(2)} \in (S_1 \cap S_2)$. 
由于$\bs{x}^{(1)}, \bs{x}^{(2)} \in (S_1 \cap S_2)$,则有$\bs{x}^{(1)}, \bs{x}^{(2)} \in S_1$且$\bs{x}^{(1)}, \bs{x}^{(2)} \in S_2$,故$\forall \lambda \in [0,1]$,都有$\lambda \bs{x}^{(1)} + (1 - \lambda) \bs{x}^{(2)} \in S_1$且$\lambda \bs{x}^{(1)} + (1 - \lambda) \bs{x}^{(2)} \in S_2$,所以$\lambda \bs{x}^{(1)} + (1 - \lambda) \bs{x}^{(2)} \in (S_1 \cap S_2)$,原命题得证. 

(2)设$S$是一凸集,$\text{int}(S)$是其内点集,任取$\bs{x}^{(1)}, \bs{x}^{(2)} \in \text{int}(S)$,$\lambda \in [0,1]$,则$\exists r > 0$,使得
\begin{align*}
    \Vert \bs{y}^{(1)} - \bs{x}^{(1)} \Vert \le r & \Rightarrow \bs{y}^{(1)} \in S, \\
    \Vert \bs{y}^{(2)} - \bs{x}^{(2)} \Vert \le r & \Rightarrow \bs{y}^{(2)} \in S
\end{align*}
可以证明内点定义中使用开球和使用闭球是等价的,这里为了讨论方便,我们使用有闭球的定义. 
要证明$\text{int}(S)$是凸集,即证明$\bs{x}^{(1)} + (1 - \lambda) \bs{x}^{(2)} \in \text{int}(S)$,也即证明$\exists \varepsilon > 0$,使得
$$
    \Vert \bs{z} - (\bs{x}^{(1)} + (1 - \lambda) \bs{x}^{(2)}) \Vert \le \varepsilon \Rightarrow \bs{z} \in S
$$
我们由三角不等式得到启发,于是上述逻辑式的前件的左式可以作如下放缩:
\begin{align*}
    & \Vert \bs{z} - (\bs{x}^{(1)} + (1 - \lambda) \bs{x}^{(2)}) \Vert \\
    = & \Vert \bs{z} - \lambda \bs{y}^{(1)} - (1 - \lambda) \bs{y}^{(2)} + \lambda (\bs{y}^{(1)} - \bs{x}^{(1)}) + (1 - \lambda) (\bs{y}^{(2)} - \bs{x}^{(2)}) \Vert \\
    \le & \Vert \bs{z} - \lambda \bs{y}^{(1)} - (1 - \lambda) \bs{y}^{(2)} \Vert + \Vert \lambda (\bs{y}^{(1)} - \bs{x}^{(1)}) \Vert + \Vert (1 - \lambda) (\bs{y}^{(2)} - \bs{x}^{(2)}) \Vert \\
    \le & \Vert \bs{z} - \lambda \bs{y}^{(1)} - (1 - \lambda) \bs{y}^{(2)} \Vert + \lambda r + (1 - \lambda) r \\
    = & \Vert \bs{z} - \lambda \bs{y}^{(1)} - (1 - \lambda) \bs{y}^{(2)} \Vert + r
\end{align*}
其中$\bs{y}^{(1)}, \bs{y}^{(2)}$满足$\Vert \bs{y}^{(i)} - \bs{x}^{(i)} \Vert \le r$,$i = 1, 2$,所以$\bs{y}^{(1)}, \bs{y}^{(2)} \in S$,故$\forall \lambda \in [0,1]$,$\bs{y}^{(1)} + (1 - \lambda) \bs{y}^{(2)} \in S$. 
我们要保证$\bs{z} \in S$,那么只需要控制$\varepsilon = r$,于是就有
$$
    0 \le \Vert \bs{z} - \lambda \bs{y}^{(1)} - (1 - \lambda) \bs{y}^{(2)} \Vert \le r - r = 0
$$
从而$\bs{z} = \lambda \bs{y}^{(1)} + (1 - \lambda) \bs{y}^{(2)} \in S$,也即待证命题:$\exists \varepsilon = r > 0$,使得
$$
    \Vert \bs{z} - (\bs{x}^{(1)} + (1 - \lambda) \bs{x}^{(2)}) \Vert \le \varepsilon \Rightarrow \bs{z} \in S
$$
的前件可推出后件,从而$\bs{x}^{(1)} + (1 - \lambda) \bs{x}^{(2)} \in \text{int}(S)$,待证命题得证. 

(3)我们先证明一个引理:$\bs{x} \in \overline{S}$当且仅当存在一个点列$\{\bs{x}^{(n)}\} \subset S$满足$\displaystyle \lim_{n \to \infty} \bs{x}^{(n)} = \bs{x}$. 
(实际上这正是聚点的另一等价定义)

先证$\Rightarrow$. 
对每一个整数$n$,取$\varepsilon_n = \frac 1n$,则能得到一个开球列$\{ B(\bs{x}, \varepsilon_n) \}$. 
应用选择公理,从这个开球列中依次取一个点,排成一个点列$\{ \bs{x} ^{(n)} \}$,则必然有
$$
    \Vert \bs{x}^{(n)} - \bs{x} \Vert = \frac{1}{n} \to 0
$$


再证$\Leftarrow$. 
由于$\displaystyle \lim_{n \to \infty} \bs{x}^{(n)} = \bs{x}$,则$\forall \varepsilon > 0$,都$\exists N > 0$,使得$\Vert \bs{x}^{(n)} - \bs{x} \Vert < \varepsilon$对于$\forall n > N$成立,从而$\forall \varepsilon > 0$,都一定存在一个$n$,使得$\bs{x}^{(n)} \in B(\bs{x}, \varepsilon)$并且$\bs{x}^{(n)} \in S$,从而$\bs{x}^{(n)} \in B(\bs{x}, \varepsilon) \cap S$,故$B(\bs{x}, \varepsilon) \cap S \neq \varnothing$,所以$\bs{x} \in \overline{S}$. 
从而引理得证. 

任取$\bs{x}, \bs{y} \in \overline{S}$,$\lambda \in [0, 1]$,由引理,$\exists \{ \bs{x}^{(n)} \}, \{ \bs{y}^{(n)} \} \subset S$,使得$\bs{x}^{(n)} \to \bs{x}$,$\bs{y}^{(n)} \to \bs{y}$. 
构造点列$\{ \lambda \bs{x}^{(n)} + (1 - \lambda) \bs{y}^{(n)} \}$,则由$\exists \{ \bs{x}^{(n)} \}, \{ \bs{y}^{(n)} \} \subset S$以及$S$是凸集,有$\{ \lambda \bs{x}^{(n)} + (1 - \lambda) \bs{y}^{(n)} \} \in S$. 
由$\bs{x}^{(n)} \to \bs{x}$,$\bs{y}^{(n)} \to \bs{y}$,知$\{ \lambda \bs{x}^{(n)} + (1 - \lambda) \bs{y}^{(n)} \} \to \lambda \bs{x} + (1 - \lambda) \bs{y}$,所以根据引理,$\lambda \bs{x} + (1 - \lambda) \bs{y} \in \overline{S}$. 
原命题得证. 

(4)暂时懒得写. 

(5)暂时懒得写. 
\proved

\begin{definition}
    {\bf(凸组合)}

    设$\bs{x}^{(1)}, \bs{x}^{(2)}, \cdots, \bs{x}^{(m)} \in \R^n$. 
    若$\forall j \in \{ 1, 2, \cdots, m \}$,$\lambda_j \ge 0$且$\displaystyle \sum_{j=1}^m \lambda_j = 1$,则称$\displaystyle \sum_{j=1}^m \lambda_j \bs{x}^{(j)}$为$\bs{x}^{(1)}, \bs{x}^{(2)}, \cdots, \bs{x}^{(m)}$的{\bf 凸组合}. 
\end{definition}

\begin{theorem}
    {\bf(凸集和凸组合的关系)}

    
    $S$是凸集$\Leftrightarrow S$中任意有限点的凸组合属于$S$. 
\end{theorem}

\prove

\proved

\begin{definition}
    {\bf(凸包)}

    设非空集合$S \subseteq \R^n$,由$S$中所有有限点的凸组合所构成的集合,被称为$S$的{\bf 凸包},记作$\text{cov}(S)$. 
\end{definition}

\begin{theorem}
    {\bf(凸集和凸包的关系)}

    如果$S$是凸集,那么$\text{cov} (S) = S$. 
\end{theorem}

\prove

\proved

\begin{definition}
    {\bf(多胞形)}

    设$\bs{x}^{(1)}, \bs{x}^{(2)}, \cdots, \bs{x}^{(m)} \in \R^n$. 
    由$\bs{x}^{(1)}, \bs{x}^{(2)}, \cdots, \bs{x}^{(m)}$的所有凸组合构成的集合称为多胞形,记作$H(\bs{x}^{(1)}, \bs{x}^{(2)}, \cdots, \bs{x}^{(m)})$. 
\end{definition}

\begin{definition}
    {\bf(单纯形)}

    设$\bs{x}^{(1)}, \bs{x}^{(2)}, \cdots, \bs{x}^{(m)} \in \R^n$. 
    若多胞形$H(\bs{x}^{(1)}, \bs{x}^{(2)}, \cdots, \bs{x}^{(m)})$满足$\bs{x}^{(2)} - \bs{x}^{(1)}, \bs{x}^{(3)} - \bs{x}^{(1)}, \cdots, \bs{x}^{(m)} - \bs{x}^{(1)}$线性无关,则称该多胞形是单纯形. 
\end{definition}

\subsubsection{凸锥、半正组合}

\begin{definition}
    {\bf(锥、凸锥)}

    设$S \subseteq \R^n$,若$\forall \bs{x} \in S$,$\lambda > 0$,必有$\lambda \bs{x} \in S$,则称$S$为以$\bs{0}$为顶点的{\bf 锥}. 
    若$S$还是凸集,则称$S$为{\bf 凸锥}.  
\end{definition}

规定$\{ \bs{0} \}, \R^n$都是凸锥. 

\begin{definition}
    {\bf(半正组合)}

    设$\bs{x}^{(1)}, \bs{x}^{(2)}, \cdots, \bs{x}^{(m)} \in \R^n$. 
    若$\displaystyle \sum_{j=1}^m \lambda_j > 0$,则称$\displaystyle \sum_{j=1}^m \lambda_j \bs{x}^{(j)}$为$\bs{x}^{(1)}, \bs{x}^{(2)}, \cdots, \bs{x}^{(m)}$的{\bf 半正组合}. 
\end{definition}

\begin{theorem}
    {\bf(凸锥和半正组合的关系)}

    $S$是凸锥$\Leftrightarrow S$中任意有限点的半正组合属于$S$. 
\end{theorem}

\prove

\proved

\subsection{凸函数}

\subsubsection{凸函数}

\subsubsection{水平集}

\subsection{凸规划}
