\section{凸集、凸函数和凸规划}

\subsection{凸集}

\subsubsection{凸集、凸组合}

\begin{definition}{凸集}{}
    设$S \subseteq \R^n$,若$\forall \bs{x}^{(1)}, \bs{x}^{(2)}\in S$,$\lambda \in [0,1]$,必有
    $$
        \lambda \bs{x}^{(1)} + (1 - \lambda) \bs{x}^{(2)} \in S
    $$
    则称$S$为{\bf 凸集}. 
\end{definition}

规定单点集$\{\bs{x}\}$和空集$\varnothing$都是空集. 
需要注意的是,方程$\bs{x} = \lambda \bs{x}^{(1)} + (1 - \lambda) \bs{x}^{(2)} = \bs{x}^{(2)} + \lambda (\bs{x}^{(1)} - \bs{x}^{(2)})$,$\lambda \in [0,1]$表示连接$\bs{x}^{(1)}$和$\bs{x}^{(2)}$的线段. 

\eg
设$\bs{A} \in \R^{m \times n}$,$\bs{b} \in \R^n$,证明以下命题成立:

(1)集合$S = \{ \bs{x} \mid \bs{Ax} = \bs{b} \}$是凸集;

(2)集合$S = \{ \bs{x} \mid \bs{Ax} \le \bs{b} \}$是凸集. 

\prove
(1)任取$\bs{x}^{(1)}, \bs{x}^{(2)} \in S$,$\lambda \in [0,1]$,则都有
\begin{align*}
    &\bs{A}(\lambda \bs{x}^{(1)} + (1 - \lambda) \bs{x}^{(2)}) \\
    =& \lambda \bs{A} \bs{x}^{(1)} + (1 - \lambda) \bs{A} \bs{x}^{(2)} \\
    =& \lambda \bs{b} + (1 - \lambda) \bs{b} \\
    =& \bs{b}
\end{align*}
从而$\lambda \bs{x}^{(1)} + (1 - \lambda) \bs{x}^{(2)} \in S$,故$S$是凸集. 

(2)任取$\bs{x}^{(1)}, \bs{x}^{(2)} \in S$,$\lambda \in [0,1]$,则都有
\begin{align*}
    &\bs{A}(\lambda \bs{x}^{(1)} + (1 - \lambda) \bs{x}^{(2)}) \\
    =& \lambda \bs{A} \bs{x}^{(1)} + (1 - \lambda) \bs{A} \bs{x}^{(2)} \\
    \le& \lambda \bs{b} + (1 - \lambda) \bs{b} \\
    =& \bs{b}
\end{align*}
从而$\lambda \bs{x}^{(1)} + (1 - \lambda) \bs{x}^{(2)} \in S$,故$S$是凸集. 
\proved

在介绍凸集的性质之前,我们先对一些将要用到的名词进行必要的说明. 

\begin{definition}{内点、内点集、开集}{}
    设$S \subseteq \R^n$,$\bs{x} \in S$,如果存在一个开球$B(\bs{x}, \varepsilon) = \{ \bs{y} \in \R^n \mid \Vert \bs{y} - \bs{x} \Vert < \varepsilon \}$,使得$B(\bs{x}, \varepsilon) \subseteq S$,则称$\bs{x}$是$S$的{\bf 内点},$S$的内点的全体称为$S$的{\bf 内点集},记作$\text{int}(S)$. 

    进一步地,如果 $\text{int}(S) = S$,则称$S$是\textbf{开集}
\end{definition}

所以$\bs{x} \in \text{int}(S) \Leftrightarrow \exists \varepsilon > 0, (\Vert \bs{y} - \bs{x} \Vert < \varepsilon \Rightarrow \bs{y} \in S)$.

\begin{definition}{聚点、闭包}{}
    设$S \subseteq \R^n$,$\bs{x} \in \R^n$,如果$\forall \varepsilon > 0$都有$B(\bs{x}, \varepsilon) \cap S \neq \varnothing$,则称$\bs{x}$是$S$的{\bf 聚点},$S$的聚点的全体称为$S$的{\bf 闭包},记作$\overline{S}$. 
\end{definition}

\begin{definition}{超平面}{}
    设$\bs{\alpha} \in \R^n \backslash \{ \bs{0} \}$,$\beta \in \R$,称集合$H = \{ \bs{x} \in \R^n \mid \bs{\alpha}^{\top} \bs{x} = \beta\}$是{\bf 超平面},简记作$H: \bs{\alpha}^{\top} \bs{x} = \beta$. 
\end{definition}

\begin{definition}{支撑超平面}{}
    设$S \subseteq \R^n$,$\bs{x}_0 \in S$,若超平面$H: \bs{\alpha}^{\top} \bs{x} = \beta$满足:
    \begin{romlist}
        \item $\exists \bs{x}^{(0)} \in S$,使得$\bs{x}^{(0)} \in H$;
        \item $\forall \bs{x} \in S$,$\bs{\alpha}^{\top} \bs{x} \le \beta$
    \end{romlist}
    则称$H$是$S$的{\bf 支撑超平面}. 
\end{definition}

\begin{definition}{分离超平面}{}
    设$S_1, S_2 \subseteq \R^n$,且$S_1 \cap S_2 = \varnothing$. 
    若存在超平面$H: \bs{\alpha}^{\top} \bs{x} = \beta$,使得$\forall \bs{x}^{(1)} \in S_1$,$\bs{\alpha}^{\top} \bs{x}^{(1)} \ge \beta$,且$\forall \bs{x}^{(2)} \in S_2$,$\bs{\alpha}^{\top} \bs{x}^{(2)} \le \beta$,则称$H$是$S_1, S_2$的{\bf 分离超平面}. 
\end{definition}

\begin{theorem}{凸集的性质}{}
    凸集有以下性质:
    \begin{enumerate}
        \item 凸集的交是凸集;
        \item 凸集的内点集是凸集;
        \item 凸集的闭包是凸集;
        \item 凸集边界上任意点存在支撑超平面;
        \item 两个互相不交的凸集之间存在分离超平面. 
    \end{enumerate}
\end{theorem}

\prove
(1)设$S_1, S_2$都是凸集,即证明$S_1 \cap S_2$是凸集,也即证明$\forall \bs{x}^{(1)}, \bs{x}^{(2)} \in (S_1 \cap S_2)$,$\forall \lambda \in [0,1]$,都有$\lambda \bs{x}^{(1)} + (1 - \lambda) \bs{x}^{(2)} \in (S_1 \cap S_2)$. 
由于$\bs{x}^{(1)}, \bs{x}^{(2)} \in (S_1 \cap S_2)$,则有$\bs{x}^{(1)}, \bs{x}^{(2)} \in S_1$且$\bs{x}^{(1)}, \bs{x}^{(2)} \in S_2$,故$\forall \lambda \in [0,1]$,都有$\lambda \bs{x}^{(1)} + (1 - \lambda) \bs{x}^{(2)} \in S_1$且$\lambda \bs{x}^{(1)} + (1 - \lambda) \bs{x}^{(2)} \in S_2$,所以$\lambda \bs{x}^{(1)} + (1 - \lambda) \bs{x}^{(2)} \in (S_1 \cap S_2)$,原命题得证. 

(2)设$S$是一凸集,$\text{int}(S)$是其内点集,任取$\bs{x}^{(1)}, \bs{x}^{(2)} \in \text{int}(S)$,$\lambda \in [0,1]$,则$\exists r > 0$,使得
\begin{align*}
    \Vert \bs{y}^{(1)} - \bs{x}^{(1)} \Vert \le r & \Rightarrow \bs{y}^{(1)} \in S, \\
    \Vert \bs{y}^{(2)} - \bs{x}^{(2)} \Vert \le r & \Rightarrow \bs{y}^{(2)} \in S
\end{align*}
可以证明内点定义中使用开球和使用闭球是等价的,这里为了讨论方便,我们使用有闭球的定义. 
要证明$\text{int}(S)$是凸集,即证明$\bs{x}^{(1)} + (1 - \lambda) \bs{x}^{(2)} \in \text{int}(S)$,也即证明$\exists \varepsilon > 0$,使得
$$
    \Vert \bs{z} - (\bs{x}^{(1)} + (1 - \lambda) \bs{x}^{(2)}) \Vert \le \varepsilon \Rightarrow \bs{z} \in S
$$
我们由三角不等式得到启发,于是上述逻辑式的前件的左式可以作如下放缩:
\begin{align*}
    & \Vert \bs{z} - (\bs{x}^{(1)} + (1 - \lambda) \bs{x}^{(2)}) \Vert \\
    = & \Vert \bs{z} - \lambda \bs{y}^{(1)} - (1 - \lambda) \bs{y}^{(2)} + \lambda (\bs{y}^{(1)} - \bs{x}^{(1)}) + (1 - \lambda) (\bs{y}^{(2)} - \bs{x}^{(2)}) \Vert \\
    \le & \Vert \bs{z} - \lambda \bs{y}^{(1)} - (1 - \lambda) \bs{y}^{(2)} \Vert + \Vert \lambda (\bs{y}^{(1)} - \bs{x}^{(1)}) \Vert + \Vert (1 - \lambda) (\bs{y}^{(2)} - \bs{x}^{(2)}) \Vert \\
    \le & \Vert \bs{z} - \lambda \bs{y}^{(1)} - (1 - \lambda) \bs{y}^{(2)} \Vert + \lambda r + (1 - \lambda) r \\
    = & \Vert \bs{z} - \lambda \bs{y}^{(1)} - (1 - \lambda) \bs{y}^{(2)} \Vert + r
\end{align*}
其中$\bs{y}^{(1)}, \bs{y}^{(2)}$满足$\Vert \bs{y}^{(i)} - \bs{x}^{(i)} \Vert \le r$,$i = 1, 2$,所以$\bs{y}^{(1)}, \bs{y}^{(2)} \in S$,故$\forall \lambda \in [0,1]$,$\bs{y}^{(1)} + (1 - \lambda) \bs{y}^{(2)} \in S$. 
我们要保证$\bs{z} \in S$,那么只需要控制$\varepsilon = r$,于是就有
$$
    0 \le \Vert \bs{z} - \lambda \bs{y}^{(1)} - (1 - \lambda) \bs{y}^{(2)} \Vert \le r - r = 0
$$
从而$\bs{z} = \lambda \bs{y}^{(1)} + (1 - \lambda) \bs{y}^{(2)} \in S$,也即待证命题:$\exists \varepsilon = r > 0$,使得
$$
    \Vert \bs{z} - (\bs{x}^{(1)} + (1 - \lambda) \bs{x}^{(2)}) \Vert \le \varepsilon \Rightarrow \bs{z} \in S
$$
的前件可推出后件,从而$\bs{x}^{(1)} + (1 - \lambda) \bs{x}^{(2)} \in \text{int}(S)$,待证命题得证. 

(3)我们先证明一个引理:$\bs{x} \in \overline{S}$当且仅当存在一个点列$\{\bs{x}^{(n)}\} \subset S$满足$\displaystyle \lim_{n \to \infty} \bs{x}^{(n)} = \bs{x}$. 
(实际上这正是聚点的另一等价定义)

先证$\Rightarrow$. 
对每一个整数$n$,取$\varepsilon_n = \frac 1n$,则能得到一个开球列$\{ B(\bs{x}, \varepsilon_n) \}$. 
应用选择公理,从这个开球列中依次取一个点,排成一个点列$\{ \bs{x} ^{(n)} \}$,则必然有
$$
    \Vert \bs{x}^{(n)} - \bs{x} \Vert = \frac{1}{n} \to 0
$$


再证$\Leftarrow$. 
由于$\displaystyle \lim_{n \to \infty} \bs{x}^{(n)} = \bs{x}$,则$\forall \varepsilon > 0$,都$\exists N > 0$,使得$\Vert \bs{x}^{(n)} - \bs{x} \Vert < \varepsilon$对于$\forall n > N$成立,从而$\forall \varepsilon > 0$,都一定存在一个$n$,使得$\bs{x}^{(n)} \in B(\bs{x}, \varepsilon)$并且$\bs{x}^{(n)} \in S$,从而$\bs{x}^{(n)} \in B(\bs{x}, \varepsilon) \cap S$,故$B(\bs{x}, \varepsilon) \cap S \neq \varnothing$,所以$\bs{x} \in \overline{S}$. 
从而引理得证. 

任取$\bs{x}, \bs{y} \in \overline{S}$,$\lambda \in [0, 1]$,由引理,$\exists \{ \bs{x}^{(n)} \}, \{ \bs{y}^{(n)} \} \subset S$,使得$\bs{x}^{(n)} \to \bs{x}$,$\bs{y}^{(n)} \to \bs{y}$. 
构造点列$\{ \lambda \bs{x}^{(n)} + (1 - \lambda) \bs{y}^{(n)} \}$,则由$\exists \{ \bs{x}^{(n)} \}, \{ \bs{y}^{(n)} \} \subset S$以及$S$是凸集,有$\{ \lambda \bs{x}^{(n)} + (1 - \lambda) \bs{y}^{(n)} \} \in S$. 
由$\bs{x}^{(n)} \to \bs{x}$,$\bs{y}^{(n)} \to \bs{y}$,知$\{ \lambda \bs{x}^{(n)} + (1 - \lambda) \bs{y}^{(n)} \} \to \lambda \bs{x} + (1 - \lambda) \bs{y}$,所以根据引理,$\lambda \bs{x} + (1 - \lambda) \bs{y} \in \overline{S}$. 
原命题得证. 

(4)暂时懒得写. 

(5)暂时懒得写. 
\proved

\begin{definition}{凸组合}{}
    设$\bs{x}^{(1)}, \bs{x}^{(2)}, \cdots, \bs{x}^{(m)} \in \R^n$. 
    若$\forall j \in \{ 1, 2, \cdots, m \}$,$\lambda_j \ge 0$且$\displaystyle \sum_{j=1}^m \lambda_j = 1$,则称$\displaystyle \sum_{j=1}^m \lambda_j \bs{x}^{(j)}$为$\bs{x}^{(1)}, \bs{x}^{(2)}, \cdots, \bs{x}^{(m)}$的{\bf 凸组合}. 
\end{definition}

\begin{theorem}{凸集和凸组合的关系}{}
    
    $S$是凸集$\Leftrightarrow S$中任意有限点的凸组合属于$S$. 
\end{theorem}

\prove
先证$\Rightarrow$. 
使用数学归纳法. 
即证明$S$中任意$m$个点的凸组合属于$S$. 
$m = 1, 2$时,由于$S$是凸集,所以显然成立. 
假设$m = k$时,结论成立,往证$m = k + 1$时结论也成立,即证明当$\displaystyle \sum_{j = 1}^{k+1} \lambda_{j} = 1$时$\displaystyle\sum_{j=1}^{k+1} \lambda_j \bs{x}_j \in S$. 
由于$m = k$时结论成立,则有:
$$
    \frac{\displaystyle\sum_{j = 1}^{k} \lambda_j \bs{x}_j}{\displaystyle\sum_{j = 1}^{k} \lambda_j} = \displaystyle\sum_{j = 1}^{k} \frac{\lambda_j}{\displaystyle\sum_{i = 1}^{k} \lambda_i} \bs{x}_j \in S
$$
从而
$$
    \displaystyle\sum_{j=1}^{k+1} \lambda_j \bs{x}^{(j)}  = \displaystyle\sum_{j = 1}^{k} \lambda_j \bs{x}^{(j)} + \lambda_{k+1} \bs{x}^{(k+1)}  = \displaystyle\sum_{j = 1}^{k} \lambda_j \cdot \frac{\displaystyle\sum_{j = 1}^{k} \lambda_j \bs{x}^{(j)}}{\displaystyle\sum_{j = 1}^{k} \lambda_j} + \lambda_{k+1} \bs{x}^{(k+1)} \in S
$$
结论得证. 

再证$\Leftarrow$. 
取$m = 2$,则若$\lambda_1, \lambda_2 > 0$且$\lambda_1 + \lambda_2 = 1$,都有$\lambda_1 \bs{x}^{(1)} + \lambda_2 \bs{x}^{(2)} \in S$,这正是凸集的定义,从而结论成立. 
\proved

\begin{definition}{凸包}{}
    设非空集合$S \subseteq \R^n$,由$S$中所有有限点的凸组合所构成的集合,被称为$S$的{\bf 凸包},记作$\text{cov}(S)$. 
\end{definition}

\begin{theorem}{凸集和凸包的关系}{}
    如果$S$是凸集,那么$\text{cov} (S) = S$. 
\end{theorem}

\prove
由定理1.8知,$\text{cov}(S) \subseteq S$,下证$S \subseteq \text{cov}(S)$. 
一个点的凸组合就是其本身,所以只要$\bs{x} \in S$,那么这个点本身的凸组合$\bs{x} \in \text{cov}(S)$,所以$S \subseteq \text{cov}(S)$. 
\proved

\begin{definition}{多胞形}{}
    设$\bs{x}^{(1)}, \bs{x}^{(2)}, \cdots, \bs{x}^{(m)} \in \R^n$. 
    由$\bs{x}^{(1)}, \bs{x}^{(2)}, \cdots, \bs{x}^{(m)}$的所有凸组合构成的集合称为多胞形,记作$H(\bs{x}^{(1)}, \bs{x}^{(2)}, \cdots, \bs{x}^{(m)})$. 
\end{definition}

\begin{definition}{单纯形}{}
    设$\bs{x}^{(1)}, \bs{x}^{(2)}, \cdots, \bs{x}^{(m)} \in \R^n$. 
    若多胞形$H(\bs{x}^{(1)}, \bs{x}^{(2)}, \cdots, \bs{x}^{(m)})$满足$\bs{x}^{(2)} - \bs{x}^{(1)}, \bs{x}^{(3)} - \bs{x}^{(1)}, \cdots, \bs{x}^{(m)} - \bs{x}^{(1)}$线性无关,则称该多胞形是单纯形. 
\end{definition}

\subsubsection{凸锥、半正组合}

\begin{definition}{锥、凸锥}{}
    设$S \subseteq \R^n$,若$\forall \bs{x} \in S$,$\lambda > 0$,必有$\lambda \bs{x} \in S$,则称$S$为以$\bs{0}$为顶点的{\bf 锥}. 
    若$S$还是凸集,则称$S$为{\bf 凸锥}.  
\end{definition}

规定$\{ \bs{0} \}, \R^n$都是凸锥. 
锥不一定包含$\bs{0}$. 

\begin{definition}{半正组合}{}
    设$\bs{x}^{(1)}, \bs{x}^{(2)}, \cdots, \bs{x}^{(m)} \in \R^n$. 
    若$\forall j \in \{ 1, 2, \cdots, m \}$,$\lambda_j \ge 0$且$\displaystyle \sum_{j=1}^m \lambda_j > 0$,则称$\displaystyle \sum_{j=1}^m \lambda_j \bs{x}^{(j)}$为$\bs{x}^{(1)}, \bs{x}^{(2)}, \cdots, \bs{x}^{(m)}$的{\bf 半正组合}. 
\end{definition}

\begin{theorem}{凸锥和半正组合的关系}{}
    $S$是凸锥$\Leftrightarrow S$中任意有限点的半正组合属于$S$. 
\end{theorem}

\prove 先证$\Rightarrow$. 
即证若$\bs{x}^{(1)}, \bs{x}^{(2)}, \cdots, \bs{x}^{(m)} \in S$. 
且$\forall j \in \{ 1, 2, \cdots, m \}$,$\lambda_j \ge 0$且$\displaystyle \sum_{j=1}^m \lambda_j > 0$,则称$\displaystyle \sum_{j=1}^m \lambda_j \bs{x}^{(j)} \in S$. 
注意到:
$$
    \sum_{j=1}^m \lambda_j \bs{x}^{(j)} 
    = \sum_{j=1}^m \lambda_j \cdot \sum_{j=1}^m \frac{\lambda_j}{\displaystyle \sum_{i=1}^m \lambda_i} \bs{x}^{(j)}
$$
其中:$\displaystyle \sum_{j=1}^m \frac{\lambda_j}{\displaystyle \sum_{i=1}^m \lambda_i} \bs{x}^{(j)}$正是$\bs{x}^{(1)}, \bs{x}^{(2)}, \cdots, \bs{x}^{(m)}$的凸组合,由定理1.8知,$\displaystyle \sum_{j=1}^m \frac{\lambda_j}{\displaystyle \sum_{i=1}^m \lambda_i} \bs{x}^{(j)} \in S$. 
此外,由$\displaystyle\sum_{j=1}^m \lambda_j > 0$以及锥的定义,有
$$
    \sum_{j=1}^m \lambda_j \cdot \sum_{j=1}^m \frac{\lambda_j}{\displaystyle \sum_{i=1}^m \lambda_i} \bs{x}^{(j)} \in S
$$
即原命题得证. 

再证$\Leftarrow$. 
显然任意有限点的凸组合一定是其半正组合,从而由定理1.8的$\Leftarrow$,$S$是凸集. 
我们仿照定理1.8的证明方法,一个点$\bs{x}$的半正组合正是$\lambda \bs{x}$,$\lambda > 0$,而“若$\forall \bs{x} \in S$,$\lambda > 0$,则$\lambda \bs{x} \in S$”正是锥的定义. 
因此原命题得证. 
\proved

\subsection{凸函数}

\subsubsection{凸函数}

\begin{definition}{凸函数}{}
    设集合$S \subseteq \R^n$为凸集,函数$f: S \to R$. 11122
    若$\forall \bs{x}^{(1)}, \bs{x}^{(2)} \in S$,$\lambda \in (0,1)$,均有
    $$
        f(\lambda \bs{x}^{(1)} + (1 - \lambda) \bs{x}^{(2)}) \le \lambda f(\bs{x}^{(1)}) + (1 - \lambda) f(\bs{x}^{(2)})
    $$
    则称$f(\bs{x})$为凸集$S$上的{\bf 凸函数}. 

    若进一步有上面不等式以严格不等式成立,则称$f(\bs{x})$为凸集$S$上的{\bf 严格凸函数}. 
    当$-f(\bs{x})$为凸函数(严格凸函数时),则称$f(\bs{x})$为{\bf 凹函数}({\bf 严格凹函数}). 
\end{definition}

\begin{theorem}{Jesen不等式}{}
    $f(\bs{x})$为凸集$S$上的凸函数的充要条件是$S$上任意有限点的凸组合的函数值不大于各点函数值的凸组合,即$\forall \bs{x}^{(1)}, \bs{x}^{(2)}, \cdots, \bs{x}^{(m)} \in S$,对于任意的序列$\{ \lambda_j \}_{j=1}^m$满足$\forall j \in \{ 1, 2, \cdots, m \}$,$\lambda_j \ge 0$且$\displaystyle \sum_{j=1}^{m} \lambda_j = 1$,都有
    $$
        f \left(\sum_{j=1}^m \lambda_j \bs{x}^{(j)} \right) \le \sum_{j=1}^m \lambda_j f(\bs{x}^{(j)})
    $$
\end{theorem}

\prove
我们仿照定理1.8进行证明. 
先证$\Rightarrow$. 
使用数学归纳法. 
$m = 1$时,待证命题显然成立. 
$m = 2$时,待证命题与凸函数定义等价,故其也成立. 
设$m = k$时,待证命题成立,往证$m = k + 1$时命题也成立. 
这里为了讨论方便,令$\displaystyle \sum_{j = 1}^k \lambda_j = \theta$,则
\begin{align*}
    f \left( \sum_{j = 1}^{k + 1} \lambda_j \bs{x}^{(j)} \right)
    & = f \left( \sum_{j = 1}^{k} \lambda_j \bs{x}^{(j)} + \lambda_{k+1} \bs{x}^{(k+1)} \right) \\
    & = f \left( \theta \sum_{j=1}^k \frac{\lambda_j}{\theta} x^{(j)} + \lambda_{k+1} \bs{x}^{(k+1)} \right) \\
    & \le \theta f \left( \frac{\lambda_j}{\theta} x^{(j)} \right) + \lambda_{k+1} f(\bs{x}^{(k+1)}) \\
    & \le \theta \left( \sum_{j=1}^k \frac{\lambda_j}{\theta} f(\bs{x}^{(j)}) \right) + \lambda_{k+1} f(\bs{x}^{(k+1)}) \\
    & = \sum_{j=1}^{k+1} \lambda_j f(\bs{x}^{(j)})
\end{align*}
命题得证. 

再证$\Leftarrow$. 
取$m = 2$时即可,因为这正是凸函数的定义. 
命题得证. 
\proved

\eg
设$f_1, f_2$是凸函数,判断:

(1)设$\lambda_1, \lambda_2 > 0$,$\lambda_1 f_1 + \lambda_2 f_2$是否为凸函数?
$\lambda_1 f_1 - \lambda_2 f_2$是否为凸函数?

(2)$f(\bs{x}) = \max \{ f_1(\bs{x}), f_2(\bs{x}) \}$是否为凸函数?
$g(\bs{x}) = \min \{ f_1(\bs{x}), f_2(\bs{x}) \}$是否为凸函数?

\solve
(1)$\lambda_1 f_1 + \lambda_2 f_2$是凸函数. 
设$f_1, f_2$定义域的交集为$S$,则$\forall \bs{x}^{(1)}, \bs{x}^{(2)} \in S$,$\forall \lambda > 0$,都有
\begin{align*}
    & \lambda_1 f_1 (\lambda \bs{x}^{(1)} + (1 - \lambda) \bs{x^{(2)}}) + \lambda_2 f_2 (\lambda \bs{x}^{(1)} + (1 - \lambda) \bs{x^{(2)}}) \\ 
    \le & \lambda_1 (\lambda f_1(\bs{x}^{(1)}) + (1 - \lambda) f_1(\bs{x^{(2)}})) + \lambda_2 (\lambda f_2(\bs{x}^{(1)}) + (1 - \lambda) f_2(\bs{x^{(2)}})) \\
    = & \lambda (\lambda_1 f_1 (\bs{x}^{(1)}) + \lambda_2 f_2 (\bs{x}^{(1)})) + (1 - \lambda) (\lambda_1 f_1 (\bs{x}^{(2)}) + \lambda_2 f_2 (\bs{x}^{(2)}))
\end{align*}

$\lambda_1 f_1 - \lambda_2 f_2$不一定是凸函数,构造反例如下:令$\lambda_1 = 1$,$\lambda_2 = 2$,$f_1(x) = f_2(x) = x^2$,从而$\lambda_1 f_1(x) - \lambda_2 f_2(x) = -x^2$在$\R$上是凹函数. 
若要构造正例,只需要调整上述反例中的$\lambda_1, \lambda_2$,使得$\lambda_1 > \lambda_2$即可. 

(2)$f(\bs{x}) = \max \{ f_1(\bs{x}), f_2(\bs{x}) \}$是凸函数. 
设$f_1, f_2$定义域的交集为$S$,则$\forall \bs{x}^{(1)}, \bs{x}^{(2)} \in S$,$\forall \lambda > 0$,都有
$$
    f_1 (\lambda \bs{x}^{(1)} + (1 - \lambda) \bs{x^{(2)}}) \le \lambda f_1(\bs{x}^{(1)}) + (1 - \lambda) f_1(\bs{x^{(2)}}) \le \lambda f(\bs{x}^{(1)}) + (1 - \lambda) f(\bs{x^{(2)}})
$$
同理,有
$$
    f_2 (\lambda \bs{x}^{(1)} + (1 - \lambda) \bs{x^{(2)}}) \le \lambda f(\bs{x}^{(1)}) + (1 - \lambda) f(\bs{x^{(2)}})
$$
从而
\begin{align*}
    & f (\lambda \bs{x}^{(1)} + (1 - \lambda) \bs{x^{(2)}}) \\
    = & \max \{ f_1 (\lambda \bs{x}^{(1)} + (1 - \lambda) \bs{x^{(2)}}), f_2 (\lambda \bs{x}^{(1)} + (1 - \lambda) \bs{x^{(2)}}) \} \\
    \le & \lambda f(\bs{x}^{(1)}) + (1 - \lambda) f(\bs{x^{(2)}})
\end{align*}

$g(\bs{x}) = \min \{ f_1(\bs{x}), f_2(\bs{x}) \}$不一定是凸函数. 
考虑如下反例:
令$f_1(x) = x^2$,$f_2(x) = (x - 1)^2$,则$g(0) = \min \{ 0, 1 \} = 0$,$g(1) = \min \{ 1, 0 \} = 0$,$g\left( \frac{1}{2} \right) = \min \left\{ \frac{1}{4}, \frac{1}{4}\right\} = \frac{1}{4}$,然而:
$$
    g \left(\frac{1}{2} \cdot 0 + \frac{1}{2} \cdot 1 \right) = g \left( \frac{1}{2} \right) = \frac{1}{4} > \frac{1}{2} g(0) + \frac{1}{2} g(1)
$$
因此$g(x)$不是凸函数. 
\solved

\begin{theorem}{凸函数的性质}{}
    设$S \subseteq \R^n$为非空凸集,函数$f: S \to \R$,则有:

    \begin{enumerate}
        \item 若$f$是凸函数,则$f$在$\text{int}(S)$上连续;
        \item 若$f$是凸函数,则其对任意方向的方向导数(若方向可行)存在;
        \item 设$S$是开集,$f$在$S$上可微,则$f$是凸函数的充要条件是$\forall \bs{x}^* \in S$,都有
        $$
            f(\bs{x}) \ge f(\bs{x}^*) + \nabla f^{\top} (\bs{x}^*) (\bs{x} - \bs{x}^*)
        $$
        $\forall \bs{x} \in S$都成立. 
        进一步地,如果$f$是严格凸函数,则其充要条件是上述不等式的严格不等式成立;
        \item 设$S$是开集,$f$在$S$上二次可微,则:
            \begin{indromlist}
                \item $f$是凸函数的充要条件是$\forall \bs{x} \in S$,$\nabla^2 f(\bs{x})$是半正定的;
                \item 若$\forall \bs{x} \in S$,$\nabla^2 f(\bs{x})$是正定的,则$f$是严格凸函数. 
            \end{indromlist}
    \end{enumerate}
\end{theorem}

\prove
(1)以下基于估计$f(\bs{x}) - f(\bs{x}^{(0)})$的上下界的证明来自知乎用户\href{https://www.zhihu.com/question/327605651/answer/703823730?share_code=ccLEatjpX2zZ&utm_psn=1974944484816082565}{@cvgmt}. 

我们先证明:凸函数$f(\bs{x})$在其定义域的内点集上处处有上界. 
设凸函数$f(\bs{x})$及其定义域为$S$,则$\forall \bs{x}^{(0)} \in \text{int}(S)$,都一定存在一个充分小的$r>0$,使得以$\bs{x}^{(0)}$为中心的正方体包含在$\text{int}(S)$中. 
由凸函数的定义
$$
    f(\lambda \bs{x}^{(1)} + (1 - \lambda) \bs{x}^{(2)}) \le \lambda f(\bs{x}^{(1)}) + (1 - \lambda) f(\bs{x}^{(2)})
$$
容易得到凸函数的最大值一定在线段端点上取得. 
反复应用上述定义,最终能够证明凸函数的最大值一定在正方形的端点处取得. 
\footnote{
    这个条件还有另一种基于Minkowski-Weyl定理(见定理\ref{Minkowski-Weyl})的证明方法. 
    由上述定理,正方体(或任意有界凸多面体)实际上是其顶点集的凸包. 之后应用Jesen不等式进行证明即可. 
}

下面考虑以$\bs{x}^{(0)}$为圆心,$r$为半径的闭球$\overline{B}(\bs{x}^{(0)}, r)$(包含于上述正方形),则由上面的证明,知道$f(\bs{x})$在该闭球上一定有上界,从而一定有上确界,记作$M_r$. 
我们先来估计$f(\bs{x}) - f(\bs{x}^{(0)})$的上界. 
在$\overline{B}(\bs{x}^{(0)}, r)$内任取一点$\bs{x}$,连接$\bs{x}^{(0)}$和$\bs{x}$并延长到球面上,交于一点$\bs{y}$,从而$\bs{x}$在以$\bs{x}^{(0)}$和$\bs{y}$为端点的线段上,则由凸函数的定义,有
$$
    f(\bs{x}) \le \frac{\Vert \bs{x} - \bs{y} \Vert}{r} f(\bs{x}^{(0)}) + \frac{\Vert \bs{x} - \bs{x}^{(0)} \Vert}{r} f(\bs{y})
$$
其中$\frac{\Vert \bs{x} - \bs{x}^{(0)} \Vert}{r} + \frac{\Vert \bs{x} - \bs{y} \Vert}{r} = 1$,从而有
\begin{align*}
    f(\bs{x}) - f(\bs{x}^{(0)}) & \le - \frac{\Vert \bs{x} - \bs{x}^{(0)} \Vert}{r} f(\bs{x}^{(0)}) + \frac{\Vert \bs{x} - \bs{y} \Vert}{r} f(\bs{y}) \\
    & = \frac{\Vert \bs{x} - \bs{x}^{(0)} \Vert}{r} (f(\bs{y}) - f(\bs{x}^{(0)})) \\
    & \le  \Vert \bs{x} - \bs{x}^{(0)} \Vert \cdot \frac{M_r - f(\bs{x}^{(0)})}{r}
\end{align*}
接下来估计$f(\bs{x}) - f(\bs{x}^{(0)})$的下界. 
类似地,在$\overline{B}(\bs{x}^{(0)}, r)$内任取一点$\bs{x}'$,连接$\bs{x}'$和$\bs{x}^{(0)}$并延长到球面上,交于一点$\bs{y}'$,从而$\bs{x^{(0)}}$在以$\bs{x'}$和$\bs{y'}$为端点的线段上,则由凸函数的定义,有
$$
    f(\bs{x}^{(0)}) \le \frac{\Vert \bs{y} - \bs{x}^{(0)} \Vert}{\Vert \bs{x} - \bs{x}^{(0)} \Vert+ r} f(\bs{x}) + \frac{\Vert \bs{x} - \bs{x}^{(0)} \Vert}{\Vert \bs{x} - \bs{x}^{(0)} \Vert+ r} f(\bs{y})
$$
其中$\frac{\Vert \bs{y} - \bs{x}^{(0)} \Vert}{\Vert \bs{x} - \bs{x}^{(0)} \Vert+ r} + \frac{\Vert \bs{x} - \bs{x}^{(0)} \Vert}{\Vert \bs{x} - \bs{x}^{(0)} \Vert+ r} = 1$,从而有:
\begin{align*}
    f(\bs{x}) - f(\bs{x}^{(0)}) & \ge \frac{\Vert \bs{x} - \bs{x}^{(0)} \Vert}{r} f(\bs{x}^{(0)}) - \frac{\Vert \bs{x} - \bs{x}^{(0)} \Vert}{r} f(\bs{y}) \\
    & = \frac{\Vert \bs{x} - \bs{x}^{(0)} \Vert}{r} (f(\bs{x}^{(0)}) - f(\bs{y})) \\
    & \ge  \Vert \bs{x} - \bs{x}^{(0)} \Vert \cdot \frac{f(\bs{x}^{(0)}) - M_r}{r}
\end{align*}
综上,有
$$
    | f(\bs{x}) - f(\bs{x}^{(0)}) | \le \Vert \bs{x} - \bs{x}^{(0)} \Vert \cdot \frac{|M_r - f(\bs{x}^{(0)})|}{r}
$$
将$r$保持为不是无穷小的固定值,从而$\frac{|M_r - f(\bs{x}^{(0)}|}{r}$为一有限值,并且令$\bs{x} \to \bs{x}^{(0)}$,从而$\Vert \bs{x} - \bs{x}^{(0)} \Vert \to 0$,于是由夹挤准则,$| f(\bs{x}) - f(\bs{x}^{(0)}) | \to 0$,于是连续性得证. 

(2)设凸函数$f(\bs{x})$的定义域为$S$,$\forall \bs{x} \in S$及其可行方向$\bs{d}$,即证明函数
$$
    g(\lambda) = \frac{f(\bs{x} + \lambda \bs{d}) - f(\bs{x})}{\lambda}
$$
在$0^+$处有定义且极限存在. 
取充分小的$\delta > 0$,由于$\bs{d}$是可行方向,故其在$(0, \delta)$上有定义. 
下证其在$0^+$处极限存在. 
取$0 < \lambda_1 < \lambda_2 < \delta$,由$f$的凸性,有:
$$
    f(\bs{x} + \lambda_1 \bs{d}) \le \frac{\lambda_1}{\lambda_2} f(\bs{x} + \lambda_2 \bs{d}) + \left( 1 - \frac{\lambda_1}{\lambda_2} \right) f(\bs{x})
$$
整理得
$$
    g(\lambda_1) = \frac{f(\bs{x} + \lambda_1 \bs{d}) - f(\bs{x})}{\lambda_1} \le \frac{f(\bs{x} + \lambda_2 \bs{d}) - f(\bs{x})}{\lambda_2} = g(\lambda_2)
$$
由$\lambda_1, \lambda_2$及$\delta$的任意性,有$g(\lambda)$在0的任意右半去心邻域内单调递增,从而极限$\displaystyle\lim_{\lambda \to 0^+} g(\lambda)$为有限数或$-\infty$. 

(3)

(4)

\proved

\eg
若$f(x) = 
\begin{cases}
    2, & x = \pm 1 ,\\
    x^2, & -1 < x < 1
\end{cases}$,
容易验证$f$是凸函数,但是$f$在边界点上不连续. 

\subsubsection{水平集}

\begin{definition}{水平集}{}
    设集合$S \subseteq \R^n$,函数$f: S \to \R$,$\alpha \in \R$,称$S_{\alpha} = \{ \bs{x} \in S \mid f(\bs{x}) \le \alpha \}$为$f(\bs{x})$在$S$上的$\alpha${\bf 水平集}. 
\end{definition}

水平集的概念相当于在地形图中,海拔高度不高于某一数值的区域. 

\begin{theorem}{凸函数的水平集是凸集}{}
    设集合$S \subseteq \R^n$是凸集,函数$f: S \to \R$是凸函数,则对$\forall \alpha \in \R$,$S_{\alpha}$是凸集. 
\end{theorem}

\prove

\proved

上述定理的逆不真. 

\eg
考虑分段函数
$f(x) = 
\begin{cases}
    1, &x \ge 0, \\
    0, &x < 0
\end{cases}
$
,函数非凸,但其任意水平集是凸集. 

\subsection{凸规划}

\begin{definition}{凸规划}{}
    \begin{enumerate}
        \item 若$(f\ S)$中的$S$为凸集,$f$是$S$上的凸函数,优化目标为$\min$时,则称$(f\ S)$为凸规划;
        \item 若$(fgh)$中的$f, g_i$为凸函数,$h_j$为线性函数,则称$(fgh)$为凸规划. 
    \end{enumerate}
\end{definition}

\begin{definition}{凸优化的最优解}{}
    设$(f\ S)$为凸优化. 
    若$\bs{x}^*$为问题$(f\ S)$的l.opt.,则$\bs{x}^*$为g.opt. 
    进一步地,若$f$是严格凸函数,则$\bs{x}^*$是$(f\ S)$的唯一g.opt. 
\end{definition}

\prove

\proved
