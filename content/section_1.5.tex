\section{凸集、凸函数和凸规划}

\begin{definition}
    {\bf(凸集)}

    设$S \subseteq \R^n$,若$\forall \bs{x}^{(1)}, \bs{x}^{(2)}\in S$,$\lambda \in [0,1]$,必有
    $$
        \lambda \bs{x}^{(1)} + (1 - \lambda) \bs{x}^{(2)} \in S
    $$
    则称$S$为{\bf 凸集}. 
\end{definition}

规定单点集$\{\bs{x}\}$和空集$\varnothing$都是空集. 
需要注意的是,方程$\bs{x} = \lambda \bs{x}^{(1)} + (1 - \lambda) \bs{x}^{(2)} = \bs{x}^{(2)} + \lambda (\bs{x}^{(1)} - \bs{x}^{(2)})$,$\lambda \in [0,1]$表示连接$\bs{x}^{(1)}$和$\bs{x}^{(2)}$的线段. 

\eg
设$\bs{A} \in \R^{m \times n}$,$\bs{b} \in \R^n$,证明以下命题成立:

(1)集合$S = \{ \bs{x} \mid \bs{Ax} = \bs{b} \}$是凸集;

(2)集合$S = \{ \bs{x} \mid \bs{Ax} \le \bs{b} \}$是凸集. 

\prove
(1)任取$\bs{x}^{(1)}, \bs{x}^{(2)} \in S$,$\lambda \in [0,1]$,则都有
\begin{align*}
    &\bs{A}(\lambda \bs{x}^{(1)} + (1 - \lambda) \bs{x}^{(2)}) \\
    =& \lambda \bs{A} \bs{x}^{(1)} + (1 - \lambda) \bs{A} \bs{x}^{(2)} \\
    =& \lambda \bs{b} + (1 - \lambda) \bs{b} \\
    =& \bs{b}
\end{align*}
从而$\lambda \bs{x}^{(1)} + (1 - \lambda) \bs{x}^{(2)} \in S$,故$S$是凸集. 

(2)任取$\bs{x}^{(1)}, \bs{x}^{(2)} \in S$,$\lambda \in [0,1]$,则都有
\begin{align*}
    &\bs{A}(\lambda \bs{x}^{(1)} + (1 - \lambda) \bs{x}^{(2)}) \\
    =& \lambda \bs{A} \bs{x}^{(1)} + (1 - \lambda) \bs{A} \bs{x}^{(2)} \\
    \le& \lambda \bs{b} + (1 - \lambda) \bs{b} \\
    =& \bs{b}
\end{align*}
从而$\lambda \bs{x}^{(1)} + (1 - \lambda) \bs{x}^{(2)} \in S$,故$S$是凸集. 
\proved

\begin{definition}
    {\bf(锥、凸锥)}

    设$S \subseteq \R^n$,若$\forall \bs{x} \in S$,$\lambda > 0$,必有$\lambda \bs{x} \in S$,则称$S$为以$\bs{0}$为顶点的{\bf 锥}. 
    若$S$还是凸集,则称$S$为{\bf 凸锥}.  
\end{definition}

规定$\{ \bs{0} \}, \R^n$都是凸锥. 

\begin{definition}
    {\bf(凸组合,半正组合)}

    设$\bs{x}^{(1)}, \bs{x}^{(2)}, \cdots, \bs{x}^{(m)} \in \R^n$. 

    (1)若$\forall j \in \{ 1, 2, \cdots, m \}$,$\lambda_j \ge 0$且$\displaystyle \sum_{j=1}^m \lambda_j = 1$,则称$\displaystyle \sum_{j=1}^m \lambda_j \bs{x}^{(j)}$为$\bs{x}^{(1)}, \bs{x}^{(2)}, \cdots, \bs{x}^{(m)}$的{\bf 凸组合}. 

    (2)若$\displaystyle \sum_{j=1}^m \lambda_j > 0$,则称$\displaystyle \sum_{j=1}^m \lambda_j \bs{x}^{(j)}$为$\bs{x}^{(1)}, \bs{x}^{(2)}, \cdots, \bs{x}^{(m)}$的{\bf 半正组合}. 
\end{definition}

\begin{theorem}
    \par\par
    $S$是凸集$\Leftrightarrow S$中任意有限点的凸组合属于$S$. 
\end{theorem}

\begin{definition}
    {\bf(凸包)}

    设非空集合$S \subseteq \R^n$,由$S$中所有有限点的凸组合所构成的集合,被称为$S$的{\bf 凸包},记作$\text{cov}(S)$. 
\end{definition}

显然,如果$S$是凸集,那么$\text{cov} (S) = S$. 
