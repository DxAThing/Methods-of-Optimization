\section{子空间、正交子空间}

\begin{definition}
    {\bf(子空间)}

    设$\bs{d}^{(i)}\in\R^n,i=1,2,\cdots,m$是线性无关的向量组,称集合$\{ \bs{x} \mid \bs{x} = \displaystyle \sum_{j=1}^m \alpha_j \bs{d}^{(j)}, \alpha_J \in \R \}$是由向量$\bs{d}^{(1)}, \bs{d}^{(2)}, \cdots, \bs{d}^{(m)}$生成的{\bf 子空间},记作
    $L(\bs{d}^{(1)}, \bs{d}^{(2)}, \cdots, \bs{d}^{(m)})$
    ,简记作$L$. 
\end{definition}

可以看到,$\forall L \subseteq \R^n$,都有$\bs{0} \in L$. 
若令$\bs{A} = (\bs{d}^{(1)}, \bs{d}^{(2)}, \cdots, \bs{d}^{(m)}) \in \R^{n \times m}$,$\bs{\alpha} = (\alpha_1, \alpha_2, \cdots, \alpha_m)^{\top} \in \R^m$,则
\begin{equation*}
    \bs{x} \in L \Leftrightarrow \exists \bs{\alpha} \in \R^m, \bs{x} = \bs{A}\bs{\alpha}
\end{equation*}
并且$\text{Rank}(\bs{A})=m$,也就是说$\bs{A}$是{\bf 列满秩}的. 
我们将会在接下来的证明中用到这个性质. 

\begin{definition}
    {\bf(正交子空间)}

    称集合$\{ \bs{x}\in\R^n \mid \bs{x}^{\top}\bs{y}=0, \forall \bs{y}\in L \}$为子空间$L$的{\bf 正交子空间},记作$L^{\bot}$. 
\end{definition}

不难看出$\forall L \subseteq \R^n$,都有$\bs{0} \in L^{\top}$. 且有
\begin{align*}
    \bs{x} \in L^{\bot}
    & \Leftrightarrow \forall \bs{y}\in L, \bs{x}^{\top}\bs{y} = 0 \\
    & \Leftrightarrow \forall \bs{\alpha} \in \R^m, \bs{x}^{\top}\bs{A}\bs{\alpha} = 0
\end{align*}

\begin{theorem}
    {\bf(子空间投影定理)}

    设$L$为$\R^n$的子空间,则$\forall \bs{z} \in \R^n$,则必有以下两个命题成立:

    (1)$\exists! \bs{x}\in L, \bs{y} \in L^{\bot}$,使得$\bs{z} = \bs{x}+\bs{y}$;

    (2)在(1)中的$\bs{x}$为问题
    \begin{align*}
        \min_{\bs{u}} \quad & \Vert \bs{z} - \bs{u} \Vert \\
        \text{s.t.} \quad & \bs{u} \in L
    \end{align*}
    的唯一解,且该问题的最优值为(1)中的$\Vert \bs{y} \Vert$.
\end{theorem}

\prove
(1)先证明存在性. 
我们从线性方程近似解的求法得到灵感. 
(2)中问题的目标实质上是求线性方程$\bs{A\alpha} = \bs{z}$在$\R^m$中的近似解,为此我们将左右两边左乘$\bs{A}^{\top}$,于是得到正规方程
\begin{equation*}
    \bs{A}^{\top}\bs{A\alpha}=\bs{A}^{\top}\bs{z}
\end{equation*}
由于$\bs{A}$是列满秩的,所以$\bs{A}^{\top}\bs{A}$是可逆的,所以$\bs{\alpha} = (\bs{A}^{\top}\bs{A})^{-1}\bs{A}^{\top}\bs{z}$,所以取
\begin{equation*}
    \bs{x} = \bs{A\alpha} = \bs{A}(\bs{A}^{\top}\bs{A})^{-1}\bs{A}^{\top}\bs{z}
\end{equation*}
接下来只需证$\bs{z} - \bs{A}(\bs{A}^{\top}\bs{A})^{-1}\bs{A}^{\top}\bs{z} \in L^{\bot}$. 
由于
\begin{align*}
    &(\bs{z} - \bs{A}(\bs{A}^{\top}\bs{A})^{-1}\bs{A}^{\top}\bs{z})^{\top}\bs{A}\bs{\beta}\\ 
    = & \bs{z}^{\top}\bs{A}\bs{\beta} - \bs{z}^{\top}\bs{A}(\bs{A}^{\top}\bs{A})^{-1}\bs{A}^{\top}\bs{A}\bs{\beta} \\
    = & \bs{z}^{\top}\bs{A}\bs{\beta} - \bs{z}^{\top}\bs{A}\bs{E}\bs{\beta} \\
    = & \bs{0}
\end{align*}
$\forall \bs{\beta}\in\R^m$都成立,所以$\bs{z} - \bs{A}(\bs{A}^{\top}\bs{A})^{-1}\bs{A}^{\top}\bs{z} \in L^{\top}$,也即取$y = \bs{z} - \bs{A}(\bs{A}^{\top}\bs{A})^{-1}\bs{A}^{\top}\bs{z} \in L^{\top}$即可. 
存在性证毕. 

接下来证明唯一性. 
假设在子空间$L$中,$\exists \bs{p} \neq \bs{x}$,使得$\exists \bs{q} \in L^{\bot}$,$\bs{z} = \bs{p}+\bs{q}$,则有$\bs{q} = \bs{x}+\bs{y}-\bs{p}$,但是
\begin{align*}
    \bs{x}^{\top}\bs{q}
    & = \bs{x}^{\top}(\bs{x}+\bs{y}-\bs{p}) \\
    & = \bs{x}^{\top}\bs{x} - \bs{x}^{\top}\bs{q} \\
    & \neq 0
\end{align*}
与$\exists \bs{q} \in L^{\bot}$矛盾,所以假设不成立,唯一性得证. 

(2)考查函数$f(t) = \Vert \bs{z} - \bs{x} - t\bs{v} \Vert^2$,$\forall \bs{v} \in L$且$\bs{v}\neq\bs{0}$,则要证明目标问题的唯一解是$\bs{x}$,即证明$f(t)$取最小值当且仅当$t=0$,也即证明$\forall v \in L$,$f'(0)=0$,且$\forall t \neq 0$,$f'(t) \neq 0$,而
\begin{align*}
    f'(t)
    &= -2 \bs{y}^{\top} \bs{v} + 2t \bs{v}^{\top} \bs{v} \\
    &= 2t \bs{v}^{\top} \bs{v}
\end{align*}
所以上述命题成立,从而目标问题的唯一解就是(1)中的$\bs{x}$,代入$\bs{u} = \bs{x}$即得到目标问题最优值为$\Vert \bs{y} \Vert$. 
\proved

\eg
设$\bs{x}, \bs{y} \in\ R^n$,若$\forall \bs{y} \in L$,都有$\bs{x}^{\top}\bs{y} \le \alpha$,则有$\bs{x} \in L^{\bot}$且$\alpha \ge 0$. 

\prove
取$\bs{y} = \bs{0}$,则$\bs{x}^{\top}\bs{y} = 0$,从而$\bs{x}^{\top}\bs{y}$的上界$\alpha \ge 0$. 
假设$\bs{x} \not\in L^{\bot}$,则由子空间投影定理的(1),$\exists \bs{x}^{(1)} \in L$,$\bs{x}^{(2)} \in L^{\bot}$,使得$\bs{x} = \bs{x}^{(1)} + \bs{x}^{(2)}$,那么
\begin{align*}
    \bs{x}^{\top}\bs{y} &= (\bs{x}^{(1)} + \bs{x}^{(2)})^{\top}\bs{y} \\
    &= \bs{x}^{(1)\top}\bs{y} + \bs{x}^{(2)\top}\bs{y} \\
    &= \bs{x}^{(1)\top}\bs{y}
\end{align*}
令$\bs{y} = \lambda \bs{x}^{(1)}$且$\lambda \to +\infty$,则$\bs{x}^{(1)\top}\bs{y} \to +\infty$,与$\bs{x}^{\top}\bs{y}$有上界$\alpha$矛盾,故必有$\bs{x} \in L^{\bot}$. 

\proved

该命题还有以下等价表述:
\begin{itemize}
    \item 若$\forall \bs{y} \in L$,都有$\bs{x}^{\top}\bs{y} \ge \alpha$,则有$\bs{x} \in L^{\bot}$且$\alpha \le 0$. 
\end{itemize}

