\section{线性规划的单纯形法}

\subsection{线性规划理论}

线性规划的单纯形法依托以下一个定理:

\begin{theorem}{解的存在性}{}
    考虑(LP)和多面体 \( S = \{ \bs{x} \in \R^n \mid\bs{Ax} = \bs{b}, x \ge 0 \} \),设 \( \bs{A} \) 满秩,\( \bs{x}^{(1)}, \bs{x}^{(2)}, \cdots, \bs{x}^{(k)} \)为所有极点,\( \bs{d}^{(1)}, \bs{d}^{(2)}, \cdots, \bs{d}^{(l)} \)为所有极方向,则
    \begin{enumerate}
        \item (LP)存在有界最优解 \( \Leftrightarrow \bs{c}^{\top}\bs{d}^{(j)} \le 0 \), \( \forall j = 1, 2, \cdots, l \);
        \item 若(LP)存在有界最优解,则最优解可以在某个极点得到. 
    \end{enumerate}
\end{theorem}

\begin{proof}
    
\end{proof}

由此,我们可以得到,我们要求(LP)的最优解,只需要求多面体有限个极点的函数值即可. 
看上去问题似乎是解决了,但是极点的个数大致随着多面体的维数成指数级增加
(参考例题1.8)!
此外,求取极点的过程伴随着矩阵的求逆,而这是非常昂贵的. 
所以我们需要一个有选择的迭代算法代替枚举算法. 

在进行相关讨论之前,由于原ppt和书中的概念过于杂乱,笔者认为有必要将线性规划相关的基础概念总结如下:

\begin{definition}{线性规划相关概念}{}
    \begin{enumerate}
        \item \textbf{基}:对于线性规划的约束条件 \( \bs{Ax} = \bs{b}, \bs{x} \ge \bs{0} \) ,设 \( \bs{B} \) 是矩阵 \( \bs{A} \) 中的一个可逆的 \( m \times m \) 的子矩阵,则称 \( \bs{B} \) 为线性规划的一个\textbf{基矩阵},简称\textbf{基},称 \( \bs{N} \) 为基矩阵 \( \bs{B} \) 对应的\textbf{非基矩阵};
        \item \textbf{基变量、非基变量}:对应于基 \( \bs{B} \) 中列向量 \( \bs{a}_{j_1}, \bs{a}_{j_2}, \cdots, \bs{a}_{j_m} \) 的变量 \( x_{j_1}, x_{j_2}, \cdots, x_{j_m} \) 称为\textbf{基变量},其他变量(即对应于非基矩阵 \( \bs{N} \) 中列向量的变量)称为\textbf{非基变量};
        \item \textbf{基本解、基本可行解、可行基}:令所有非基变量为0,可以得到 \( m \) 个关于基变量 \( x_{j_1}, x_{j_2}, \cdots, x_{j_m} \) 的线性方程组,解这个线形方程组得到基变量的值,称如上得到的解为一个\textbf{基本解};若得到的基变量的值均非负,则称其为\textbf{基本可行解},称这个基 \( \bs{B} \) 为\textbf{可行基}. 写成矩阵形式,
        \[
            \bs{x} = 
            \begin{pmatrix}
                \bs{x}_{\bs{B}} \\ \bs{x}_{\bs{N}}
            \end{pmatrix}
            =
            \begin{pmatrix}
                \bs{B}^{-1}\bs{b} \\ \bs{0}
            \end{pmatrix}
        \]
        称为线性规划问题的一个\textbf{基本可行解}. 
    \end{enumerate}
\end{definition}

显然必有以下定理成立:

\begin{theorem}{线性规划基本定理}{}
    线性规划的基本可行解就是可行域的极点. 
\end{theorem}

下面的定理表明了我们不需要将所有的极点都代入目标函数,也可以精确地求出最优解opt..

\begin{theorem}{最优性定理}{}
    考虑(LP),条件同上,设 \( \bs{x}^* \) 为极点,存在基分解 \( \bs{A} = (\bs{B}, \bs{N}) \),其中 \( \bs{B} \) 为 \( m \) 阶可逆矩阵,使 \(\bs{x}^* = (\bs{x}_{\bs{B}}^{*\top}, \bs{x_N}^{*\top})^{\top}\) ,此处分解与 \( \bs{A} \) 相同,且 \( \bs{x}^*_{\bs{B}} = \bs{B}^{-1} \bs{b} \ge 0 \) , \( \bs{x}^*_{\bs{N}} = 0 \). 
    相应 \( \bs{c}^{\top} = (\bs{c}^{\top}_{\bs{B}}, \bs{c}^{\top}_{\bs{N}})^{\top} \) ,则
    \begin{enumerate}
        \item 若 \( \bs{c}_{\bs{N}}^{\top} - \bs{c}_{\bs{B}}^{\top} \bs{B}^{-1} \bs{N} \le 0 \),则 \( \bs{x}^* \)为opt.;
        \item 若存在属于 \( \bs{N} \) 的分量 \( j \) ,使得\( \bs{c}_j - \bs{c}^{\top}_{\bs{B}}{\bs{B}}^{-1} \bs{a}_j > 0 \) ,且 \( \bs{B}^{-1} \bs{a}_j \le 0 \) ,则(LP)无有界解. 
    \end{enumerate}
\end{theorem}

\begin{proof}
    
\end{proof}



\subsection{表格单纯形法原理及算法原理}

让我们回忆线性规划的标准形式如下
\[
    \text{(LP)} \quad 
    \begin{aligned}
        \max & \quad  z = \bs{c}^{\top}\bs{x} \\
        \text{s.t.} & \quad \bs{Ax} = \bs{b} \\
        & \quad \bs{x} \ge \bs{0}
    \end{aligned}
\]
其中, \( \bs{c}, \bs{x} \in \R^n \) ,\( \bs{b} \in \R^m \) ,\( \bs{A} \) 为 \( m \times n \) 行满秩矩阵,即\( \text{rank}(\bs{A}) = m \). 

\subsubsection{单纯形法原理及算法过程}

下边的算法介绍了单纯形法求线性规划最优解的过程. 

\begin{algorithm}[H]
    \caption{单纯形法迭代算法 (Simplex Method)}
    \SetKwInOut{Input}{输入}
    \SetKwInOut{Output}{输出}
    \SetKwInOut{Init}{初始化}
    \Input{系数矩阵 \( \bs{A} \in \mathbb{R}^{m \times n} \),向量 \( \bs{b} \in \mathbb{R}^m \),价值向量 \( \bs{c} \in \mathbb{R}^n \)}
    \Output{最优解 \( \bs{x}^* \) 或问题无界结论}
    \Init{已知极点 \( \bs{x}^{(k)} \),基分解 \( \bs{A} = (\bs{B}, \bs{N}) \),满足 \( \bs{x}_B^{(k)} = \bs{B}^{-1}\bs{b} \ge 0 \),\( \bs{x}_N^{(k)} = \bs{0} \)}
    \BlankLine
    \nl 计算检验向量:\( \bs{\sigma}_N^{\text{T}} = \bs{c}_N^{\text{T}} - \bs{c}_B^{\text{T}} \bs{B}^{-1} \bs{N} \);\\
    \nl \If{\( \bs{\sigma}_N \le \bs{0} \)}{
        \Return \( \bs{x}^{(k)} \) 为opt.;\\
    }
    \nl 选择进基变量索引 \( j \) 使得 \( \sigma_j > 0 \);\\
    \nl \If{\( \bs{B}^{-1} \bs{a}_j \le \bs{0} \)}{
        \Return 问题无有界解;\\
    }
    \nl \Else{
        计算步长 \( \alpha = \min \left\{ \frac{(\bs{B}^{-1}\bs{b})_i}{(\bs{B}^{-1}\bs{a}_j)_i} \mid (\bs{B}^{-1}\bs{a}_j)_i > 0 \right\} = \frac{(\bs{B}^{-1}\bs{b})_r}{(\bs{B}^{-1}\bs{a}_j)_r} \);\\
        更新解向量:\( \bs{x}^{(k+1)} = \bs{x}^{(k)} + \alpha \bs{d} \);\\
        \Indp
            其中 \( \bs{d}^{\text{T}} = (\bs{d}_B^{\text{T}}, \bs{d}_N^{\text{T}}) \),且 \( \bs{d}_B = -\bs{B}^{-1}\bs{a}_j\),\( \bs{d}_N = \bs{e}_j \);\\
        \Indm
        \nl 更新基矩阵 \( \bs{B} \) 为 \( \bs{B}' \),令 \( k \leftarrow k+1 \) 并返回步骤 1;\\
    }
\end{algorithm}

为了说明这个算法的合理性,我们需要说明以下两点成立:

\begin{theorem}{}{}
    上述算法中,更新点 \( \bs{x}^{(k+1)} \)
    \begin{enumerate}
        \item 是更优的;
        \item 也是 \( S \) 的一个极点. 
    \end{enumerate}
\end{theorem}

\begin{proof}
    
\end{proof}

\subsubsection{单纯形表}

设 \( \bs{x} \) 为初始极点,相应分解 \( \bs{A} = (\bs{B}, \bs{N}) \),则列单纯形表如下:
\begin{longtblr}[
    label = none,
    entry = none,
]{
    cells = {c},
    vline{2-3,5-6} = {2-3}{},
    hline{2-4} = {2-5}{},
}
 & \( f \) & \( \bs{x}_{\bs{B}}^{\top} \) & \( \bs{x}_{\bs{N}}^{\top} \) & RHS &  \\
目标行 & 1 & \( \bs{c}_{\bs{B}}^{\top} \) & \( \bs{c}_{\bs{N}}^{\top} \) & 0 & 1行 \\
约束行 & 0 & \( \bs{B} \) & \( \bs{N} \) & \( \bs{b} \) & \( m \)行 \\
 & 1列 & \( m \)行 & \( n-m \)行 & 1列 &  
\end{longtblr}

\noindent
作变换,使前 \( m+1 \) 列对应的 \( m+1 \)阶矩阵变为单位矩阵,相当于该表左乘
\( 
\begin{pmatrix}
    1 & \bs{c}_{\bs{B}}^{\top} \\
    0 & \bs{B}
\end{pmatrix}^{-1}
=
\begin{pmatrix}
    1 & -\bs{c}_{\bs{B}}^{\top} \bs{B}^{-1} \\
    0 & \bs{B}^{-1}
\end{pmatrix} \). 
消去方法为Gauss主元消去法. 
于是得到
\begin{longtblr}[
    label = none,
    entry = none,
]{
    cells = {c},
    vline{2-3,5-6} = {2-3}{},
    hline{2-4} = {2-5}{},
}
 & \( f \) & \( \bs{x}_{\bs{B}}^{\top} \) & \( \bs{x}_{\bs{N}}^{\top} \) & RHS &  \\
目标行 & 1 & \( \bs{0}^{\top} \) & \( \bs{c}_{\bs{N}}^{\top} - \bs{c}_{\bs{B}}^{\top}\bs{B}^{-1}\bs{N} \) & \( -\bs{c}_{\bs{B}}^{\top}\bs{B}^{-1}\bs{b} \) & 1行 \\
约束行 & 0 & \( \bs{I} \) & \( \bs{B}^{-1}\bs{N} \) & \( \bs{B}^{-1}\bs{b} \) & \( m \)行 \\
 & 1列 & \( m \)行 & \( n-m \)行 & 1列 &  
\end{longtblr}

\noindent
根据算法内容,令\textbf{检验向量} \( \bs{\sigma}^{\top}_{\bs{N}} = \bs{c}_{\bs{N}}^{\top} - \bs{c}_{\bs{B}}^{\top}\bs{B}^{-1}\bs{N} \),则单纯形表转化为
\begin{longtblr}[
    label = none,
    entry = none,
]{
    cells = {c},
    vline{2-3,5-6} = {2-3}{},
    hline{2-4} = {2-5}{},
}
 & \( f \) & \( \bs{x}_{\bs{B}}^{\top} \) & \( \bs{x}_{\bs{N}}^{\top} \) & RHS &  \\
目标行 & 1 & \( \bs{0}^{\top} \) & \( \bs{\sigma}^{\top} \) & \( -z = - \bs{c}^{\top} \bs{x}^* \) & 1行 \\
约束行 & 0 & \( \bs{I} \) & \( \bs{B}^{-1}\bs{N} \) & \( \bs{x}^*_{\bs{B}} \) & \( m \)行 \\
 & 1列 & \( m \)行 & \( n-m \)行 & 1列 &  
\end{longtblr}

\noindent
可是我们知道,对矩阵求逆的代价是十分昂贵的(对一个 \( n \) 阶矩阵求逆的时间复杂度为 \( O(n^3) \)). 
所以如果我们能使 \( \bs{B} = \bs{I} \) ,则 \( \bs{B}^{-1} = \bs{I} \) ,此时求矩阵 \( \bs{B} \) 的逆将会很简单,此时
\begin{longtblr}[
    label = none,
    entry = none,
]{
    cells = {c},
    vline{2-3,5-6} = {2-3}{},
    hline{2-4} = {2-5}{},
}
 & \( f \) & \( \bs{x}_{\bs{B}}^{\top} \) & \( \bs{x}_{\bs{N}}^{\top} \) & RHS &  \\
目标行 & 1 & \( \bs{0}^{\top} \) & \( \bs{c}_{\bs{N}}^{\top} - \bs{c}_{\bs{B}}^{\top}\bs{N} \) & \( -\bs{c}_{\bs{B}}^{\top}\bs{b} \) & 1行 \\
约束行 & 0 & \( \bs{I} \) & \( \bs{N} \) & \( \bs{b} \) & \( m \)行 \\
 & 1列 & \( m \)行 & \( n-m \)行 & 1列 &  
\end{longtblr}

\noindent
综上所述,我们将单次迭代所需要用到的所有元素集中在一张表中,就得到了如下经过改造的单纯形表:(其中,表\ref{unmodified}对矩阵 \( \bs{B} \) 不作除可逆外的任何要求,表\ref{modified}中令 \( \bs{B} = \bs{I} \) )

\begin{table}[ht]
    \centering
    \caption{经过改造后的单纯形表}
    \label{unmodified}
    \begin{tblr}{
        % width = 0.9\linewidth,
        colspec = {X[c] X[c] X[c] X[c] X[c] X[c]}, 
        cells = {m}, % 内容垂直居中
        cell{1}{1,2,3,6} = {r=2}{}, % 合并前三列和最后一列的行
        cell{4}{1} = {c=2}{},
        % 设置纵线
        vline{1-4,6-7} = {1-4}{solid}, 
        % vline{5} = {1-2}{solid}, % c_B^T 和 c_N^T 之间的竖线
        % 设置横线
        hline{1,3,5} = {1-6}{solid}, % 顶部、中部、底部横线
        hline{2} = {4-5}{solid},     % 仅在 c 向量和 x 向量之间画横线
        hline{4} = {1-6}{solid},     % 表格中间的横线
    }
    \( \bs{c}_{\bs{B}} \) & \( \bs{x}_{\bs{B}} \) & \( \bs{b} \) & \( \bs{c}_{\bs{B}}^{\top} \) & \( \bs{c}_{\bs{N}}^{\top} \) & \( \bs{\alpha} \) \\
    &  &  & \( \bs{x}_{\bs{B}}^{\top} \) & \( \bs{x}_{\bs{N}}^{\top} \) &  \\
    \( * \) & \( * \) & \( \bs{B}^{-1}\bs{b} \) & \( \bs{I} \) & \( \bs{B}^{-1}\bs{N} \) & \( * \) \\
    \( -z \) &  &  \( -\bs{c}_{\bs{B}}^{\top}\bs{B}^{-1}\bs{b} \) & \( \bs{0}^{\top} \) & \( \bs{c}_{\bs{N}}^{\top} - \bs{c}_{\bs{B}}^{\top}\bs{B}^{-1}\bs{N} \) &  
    \end{tblr}
\end{table}

\begin{table}[ht]
    \centering
    \caption{经过改造后的单纯形表(令 \( \bs{B} = \bs{I} \))}
    \label{modified}
    \begin{tblr}{
        colspec = {X[c] X[c] X[c] X[c] X[c] X[c]}, 
        cells = {m}, % 内容垂直居中
        cell{1}{1,2,3,6} = {r=2}{}, % 合并前三列和最后一列的行
        cell{4}{1} = {c=2}{},
        % 设置纵线
        vline{1-4,6-7} = {1-4}{solid}, 
        % vline{5} = {1-2}{solid}, % c_B^T 和 c_N^T 之间的竖线
        % 设置横线
        hline{1,3,5} = {1-6}{solid}, % 顶部、中部、底部横线
        hline{2} = {4-5}{solid},     % 仅在 c 向量和 x 向量之间画横线
        hline{4} = {1-6}{solid},     % 表格中间的横线
    }
    \( \bs{c}_{\bs{B}} \) & \( \bs{x}_{\bs{B}} \) & \( \bs{b} \) & \( \bs{c}_{\bs{B}}^{\top} \) & \( \bs{c}_{\bs{N}}^{\top} \) & \( \bs{\alpha} \) \\
    &  &  & \( \bs{x}_{\bs{B}}^{\top} \) & \( \bs{x}_{\bs{N}}^{\top} \) &  \\
    \( * \) & \( * \) & \( \bs{b} \) & \( \bs{I} \) & \( \bs{N} \) & \( * \) \\
    \( -z \) &  & \( -\bs{c}_{\bs{B}}^{\top}\bs{b} \) & \( \bs{0}^{\top} \) & \( \bs{c}_{\bs{N}}^{\top} - \bs{c}_{\bs{B}}^{\top}\bs{N} \) &
    \end{tblr}
\end{table}

该表的填写规则如下:(合并的单元格按照合并之前最左上角的单元格编号)
\begin{itemize}
    \item 表格第3行第1-2列(即\( \bs{c}_{\bs{B}}\) 和 \(\bs{x}_{\bs{B}} \)所示单元格下边的单元格),将列向量 \( \bs{c}_{\bs{B}}, \bs{x}_{\bs{B}} \) 分别展开后写入;
    \item 表格第3行第3列(即\( \bs{B}^{-1}\bs{b} \) 或 \( \bs{b} \) 所示单元格),将列向量 \( \bs{B}^{-1}\bs{b} \) 或 \( \bs{b} \) 展开后写入;
    \item 表格第4行第3列(即\( -\bs{c}_{\bs{B}}^{\top}\bs{B}^{-1}\bs{b} \) 或 \( -\bs{c}_{\bs{B}}^{\top}\bs{b}\) 所示单元格),将标量 \( -\bs{c}_{\bs{B}}^{\top}\bs{B}^{-1}\bs{b} \) 或 \( -\bs{c}_{\bs{B}}^{\top}\bs{b} \) 计算出来后写入;
    \item 表格第4-5列,将所示行向量或矩阵展开后写入;
    \item 表格第3行第6列(即 \( \bs{\alpha} \) 所示单元格下边的单元格),将列向量 \( \bs{\alpha} \) 展开后写入. 
    其中,步长 \( \alpha_i = 
    \begin{cases}
        \frac{b_i}{a_{ik}}, & a_{ik}>0 \\
        \infty, & a_{ik} \le 0
    \end{cases} \), \( k \) 为进基变量索引. 
\end{itemize}

\subsubsection{引入单位阵 \( \bs{I} \) }

这一小节我们分为两种情况来讨论:对\textbf{规范形式}求解线性规划问题和\textbf{一般情况下}的求解线性规划问题. 

考虑线性规划的规范形式(P)
\begin{align*}
    \max & \quad z = c_1x_1 + c_2 x_2 + \cdots + c_n x_n \\
    \text{s.t.} & \quad a_{11} x_1 + a_{12} x_2 + \cdots + a_{1n} x_n \le b_1 \\
    & \quad a_{21} x_1 + a_{22} x_2 + \cdots + a_{2n} x_n \le b_2 \\
    & \quad \vdots \\
    & \quad a_{m1} x_1 + a_{m2} x_2 + \cdots + a_{mn} x_n \le b_m \\
    & \quad x_1, x_2, \cdots, x_n \ge 0 \\
    & \quad b_1, b_2, \cdots, b_m > 0
\end{align*}
加入松弛变量 \( {x}_{n+1}, {x}_{n+2}, \cdots, {x}_{n+m} \) 后,转化为标准形式
\begin{align*}
    \max & \quad z = c_1x_1 + c_2 x_2 + \cdots + c_n x_n \\
    \text{s.t.} & \quad a_{11} x_1 + a_{12} x_2 + \cdots + a_{1n} x_n + x_{n+1} = b_1 \\
    & \quad a_{21} x_1 + a_{22} x_2 + \cdots + a_{2n} x_n + x_{n+2} = b_2 \\
    & \quad \vdots \\
    & \quad a_{m1} x_1 + a_{m2} x_2 + \cdots + a_{mn} x_n + x_{n+m} = b_m \\
    & \quad x_1, x_2, \cdots, x_n, {x}_{n+1}, {x}_{n+2}, \cdots, {x}_{n+m} \ge 0 \\
    & \quad b_1, b_2, \cdots, b_m > 0
\end{align*}
我们可以得到,\( x_j = 0 \),\( j = 1, 2, \cdots, n \);\( x_{n+1} = b_i \),\( i = 1, 2, \cdots, m \) 是 \( S \) 的一个基本可行解 ,对应的 \( \bs{B} \) 是单位矩阵. 

应用单纯形法解线性规划问题有如下注意事项:
\begin{itemize}
    \item 每一步运算只能用矩阵初等行变换;
    \item 表中第3列的数总应保持非负;
    \item 当所有检验数均非正时,得到最优单纯形表;
    \item 检验数的更新也相当于是主元消去. 
\end{itemize}

\eg
用单纯形法解如下线性规划问题:
\begin{enumerate}
    \item \( \begin{aligned}[t]
            \max & \quad z = 1500x_1 + 2500 x_2 \\
            \text{s.t.} & \quad 3 x_1 + 2 x_2 + x_3 = 65 \\
            & \quad 2 x_1 + x_2 + x_4 = 40 \\
            & \quad 3 x_2 + x_5 = 75 \\
            & \quad x_1, x_2, \cdots, x_5 \ge 0
        \end{aligned} 
    \)
    \item \( \begin{aligned}[t]
            \max & \quad z = 1500x_1 + 1000 x_2 \\
            \text{s.t.} & \quad 3 x_1 + 2 x_2 + x_3 = 65 \\
            & \quad 2 x_1 + x_2 + x_4 = 40 \\
            & \quad 3 x_2 + x_5 = 75 \\
            & \quad x_1, x_2, \cdots, x_5 \ge 0
        \end{aligned} 
    \)
\end{enumerate}

\begin{solution}
    
\end{solution}

接下来我们讨论一般情况,即初始基本可行解不明显时的情况. 

考虑一般问题
\begin{align*}
    \max & \quad z = c_1x_1 + c_2 x_2 + \cdots + c_n x_n \\
    \text{s.t.} & \quad a_{11} x_1 + a_{12} x_2 + \cdots + a_{1n} x_n = b_1 \\
    & \quad a_{21} x_1 + a_{22} x_2 + \cdots + a_{2n} x_n = b_2 \\
    & \quad \vdots \\
    & \quad a_{m1} x_1 + a_{m2} x_2 + \cdots + a_{mn} x_n = b_m \\
    & \quad x_1, x_2, \cdots, x_n \ge 0
\end{align*}
其中右端项 \( b_1, b_2, \cdots, b_m \ge 0 \) .
接下来介绍两种方法. 

\textbf{大 \( M\) 法}:

引入人工变量 \( x_{n+i} \ge 0 \) , \( i = 1, 2, \cdots, m \) 及充分大正数 \( M \),得到
\begin{align*}
    \max & \quad z = c_1x_1 + c_2 x_2 + \cdots + c_n x_n - Mx_{n+1} - \cdots - Mx_{n+m} \\
    \text{s.t.} & \quad a_{11} x_1 + a_{12} x_2 + \cdots + a_{1n} x_n + x_{n+1} = b_1 \\
    & \quad a_{21} x_1 + a_{22} x_2 + \cdots + a_{2n} x_n + x_{n+2} = b_2 \\
    & \quad \vdots \\
    & \quad a_{m1} x_1 + a_{m2} x_2 + \cdots + a_{mn} x_n + x_{n+m} = b_m \\
    & \quad x_1, x_2, \cdots, x_n, x_{n+1}, \cdots, x_{n+m} \ge 0
\end{align*}
显然,\( x_j = 0 \),\( j = 1, 2, \cdots, n \);\( x_{n+1} = b_i \),\( i = 1, 2, \cdots, m \) 是 \( S \) 的一个基本可行解 ,对应的 \( \bs{B} \) 是单位矩阵. 
若得到的最优解满足 \( x_{n+i} = 0 \),\( i = 1, 2, \cdots, m \),则其为原问题的\textbf{最优解};否则,原问题\textbf{无可行解}. 

\eg
应用大 \( M\) 法解如下线性规划问题
\begin{align*}
    \max & \quad z = 5x_1 + 2x_2 + 3x_3 - x_4 \\
    \text{s.t.} & \quad x_1 + 2x_2 + 3x_3 = 15 \\
    & \quad 2x_1 + x_2 + 5x_3 = 20 \\
    & \quad x_1 + 2x_3 + 4x_3 + x_4 = 26 \\
    & \quad x_1, x_2, x_3, x_4 \ge 0 
\end{align*}

\begin{solution}
    
\end{solution}

\textbf{两阶段法}:

两阶段法的核心思想是:先求出原问题的一个基本可行解(第一阶段),再求解原问题(第二阶段). 

第一阶段中,引入人工变量 \( x_{n+i} \ge 0 \) , \( i = 1, 2, \cdots, m \) ,构造:
\begin{align*}
    \max & \quad z = - x_{n+1} - x_{n+2} -  \cdots - x_{n+m} \\
    \text{s.t.} & \quad a_{11} x_1 + a_{12} x_2 + \cdots + a_{1n} x_n + x_{n+1} = b_1 \\
    & \quad a_{21} x_1 + a_{22} x_2 + \cdots + a_{2n} x_n + x_{n+2} = b_2 \\
    & \quad \vdots \\
    & \quad a_{m1} x_1 + a_{m2} x_2 + \cdots + a_{mn} x_n + x_{n+m} = b_m \\
    & \quad x_1, x_2, \cdots, x_n, x_{n+1}, \cdots, x_{n+m} \ge 0
\end{align*}
之后求解上述问题. 显然,\( x_j = 0 \),\( j = 1, 2, \cdots, n \);\( x_{n+1} = b_i \),\( i = 1, 2, \cdots, m \) 是 \( S \) 的一个基本可行解 ,对应的 \( \bs{B} \) 是单位矩阵. 
若得到的最优解满足 \( x_{n+i} = 0 \),\( i = 1, 2, \cdots, m \),则其为原问题的\textbf{基本可行解};否则,原问题\textbf{无可行解}. 
得到原问题的基本可行解之后,求解原问题. 

\eg
应用两阶段法解如下线性规划问题
\begin{align*}
    \max & \quad z = 5x_1 + 2x_2 + 3x_3 - x_4 \\
    \text{s.t.} & \quad x_1 + 2x_2 + 3x_3 = 15 \\
    & \quad 2x_1 + x_2 + 5x_3 = 20 \\
    & \quad x_1 + 2x_3 + 4x_3 + x_4 = 26 \\
    & \quad x_1, x_2, x_3, x_4 \ge 0 
\end{align*}

\begin{solution}
    
\end{solution}
