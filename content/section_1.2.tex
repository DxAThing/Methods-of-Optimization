\section{基本概念和符号}

接下来将约定一些术语的符号表示:
\begin{itemize}
    \item $\R^n$——$n$维欧式空间;
    \item $\bs{x} = (x_1, x_2, \cdots, x_n)^{\top} \in \R^n$——表示$\R^n$中的一个点或一个向量,其中分量$x_i\in\R$;
    \item $\bs{d} = (d_1, d_2, \cdots, d_n)^{\top} \in \R^n$,其中$\bs{d} \neq \bs{0}$——表示从$\bs{0}$指向$\bs{d}$的方向. 
    此外,设$\bs{d}^{(1)}, \bs{d}^{(2)} \in \R^n$,若$\exists \lambda > 0$,使得$\bs{d}^{(1)} = \lambda \bs{d}^{(2)}$,则称$\bs{d}^{(1)}, \bs{d}^{(2)}$是{\bf 同方向}的. 
    \item $\bs{x}+\lambda \bs{d}$——表示从$\bs{x}$点出发,沿$d$方向移动$\lambda \bs{d}$长度得到的点.
\end{itemize}

下面是对欧式空间中向量的运算的定义. 

\begin{definition}{向量的运算}{}
    设$\bs{x} = (x_1, x_2, \cdots, x_n)^{\top} \in \R^n$,$\bs{y} = (y_1, y_2, \cdots, y_n)^{\top} \in \R^n$,则

    \begin{enumerate}
        \item {\bf 向量的内积运算}:$\bs{x}^{\top} \bs{y} = \bs{y}^{\top} \bs{x} = \displaystyle \sum_{i=1}^n x_iy_i$;
        \item {\bf 向量长度}:$\| \bs{x} \| = \sqrt{\bs{x}^{\top} \bs{x}} = \sqrt{\displaystyle \sum_{i=1}^n x_i^2}$;
        \item {\bf 两点间距离}:$\| \bs{x} - \bs{y} \|$. 
    \end{enumerate}
\end{definition}

\begin{theorem}{三角不等式}{}
    $\forall \bs{x}, \bs{y} \in \R^n$,有
    \begin{equation}
        \| \bs{x}+\bs{y} \| \le \| \bs{x} \| + \| \bs{y} \|
    \end{equation}
\end{theorem}

\prove
对左式和右式分别平方后相减,有
\begin{align*}
    \text{左式}^2 - \text{右式}^2 
    & = (\bs{x}+\bs{y})^{\top}(\bs{x}+\bs{y}) - (\bs{x}^{\top}\bs{x} + \bs{y}^{\top}\bs{y} + 2 \| \bs{x} \| \| \bs{y} \|) \\
    & = (\bs{x}^{\top}\bs{x} + \bs{y}^{\top}\bs{y} + 2\bs{x}^{\top}\bs{y}) - (\bs{x}^{\top}\bs{x} + \bs{y}^{\top}\bs{y} + 2 \| \bs{x} \| \| \bs{y} \|) \\
    &= 2(\bs{x}^{\top}\bs{y} - \| \bs{x} \| \| \bs{y} \|)
\end{align*}
也即只需证明柯西不等式
\begin{equation*}
    \bs{x}^{\top}\bs{y} \le \| \bs{x} \| \| \bs{y} \|
\end{equation*}
成立即可. 
于是我们构造关于$t$的方程
\begin{equation*}
    \| \bs{x} +t\bs{y} \|^2 = \bs{y}^{\top}\bs{y}t^2 + 2 \bs{x}^{\top}\bs{y}t + \bs{x}^{\top}\bs{x} = 0
\end{equation*}
显然由于向量的模长一定非负,故该方程至多有一个解,也即该二次方程的判别式
\begin{equation*}
    \Delta =  4 (\bs{x}^{\top}\bs{y})^2 - 4 (\bs{x}^{\top}\bs{x})(\bs{y}^{\top}\bs{y}) \le 0
\end{equation*}
从而柯西不等式得证,从而原不等式得证. 
\proved

\begin{definition}{点列的收敛}{}
    设$\{\bs{x}^{(n)}\}$是定义在$\R^n$上的点列,若$\exists \bs{x} \in\R^n$,使得$\displaystyle\lim_{n\to\infty}\| \bs{x}^{(n)} - \bs{x} \| = 0$,则称点列$\{\bs{x}^{(n)}\}$收敛到$\bs{x}$,记作
    \begin{equation*}
        \lim_{n\to\infty} \bs{x}^{(n)} = \bs{x}
    \end{equation*}
\end{definition}

\begin{definition}{向量的大小关系}{}
    设$\bs{x}, \bs{y} \in \R^n$,若$\forall i \in \{ 1, 2, \cdots ,n \}$,都有$x_i \le y_i$,则称$\bs{x} \le \bs{y}$. 
    类似可规定$\bs{x} \ge \bs{y}$,$\bs{x} < \bs{y}$,$\bs{x} > \bs{y}$. 
\end{definition}

显然,在如上定义中,$(\R^n, \le)$并不是全序集(证明略). 
特别地,$\bs{x} \le \bs{0} \Leftrightarrow \forall i \in \{ 1, 2, \cdots ,n \}, x_i \le 0$. 

\begin{theorem}{}{}
    设$\bs{x}, \bs{y} \in \R^n$,若$\forall \bs{y} \ge \bs{0}$,都有$\bs{x}^{\top}\bs{y} \le \alpha$,则有$\bs{x} \le \bs{0}$且$\alpha \ge 0$. 
\end{theorem}

\prove
取$\bs{y} = \bs{0}$,则$\bs{x}^{\top}\bs{y} = 0$,从而$\bs{x}^{\top}\bs{y}$的上界$\alpha \ge 0$. 
假设$\bs{x} \le 0$不成立,则$\exists i \in \{1, 2, \cdots, n\}$,使得$x_i > 0$. 
令$y_i \to +\infty$,则有$\bs{x}^{\top}\bs{y} \to +\infty$,从而与$\bs{x}^{\top}\bs{y}$有上界$\alpha$矛盾,故必有$\bs{x} \le \bs{0}$. 
\proved

该命题还有以下等价表述:
\begin{itemize}
    \item 若$\forall \bs{y} \le \bs{0}$,都有$\bs{x}^{\top}\bs{y} \le \alpha$,则有$\bs{x} \ge \bs{0}$且$\alpha \ge 0$;
    \item 若$\forall \bs{y} \ge \bs{0}$,都有$\bs{x}^{\top}\bs{y} \ge \alpha$,则有$\bs{x} \ge \bs{0}$且$\alpha \le 0$;
    \item 若$\forall \bs{y} \le \bs{0}$,都有$\bs{x}^{\top}\bs{y} \ge \alpha$,则有$\bs{x} \le \bs{0}$且$\alpha \le 0$. 
\end{itemize}
