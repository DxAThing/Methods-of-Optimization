% 定义本书中使用到的格式

\ctexset {chapter = {
  pagestyle = headings
  }
}
\ctexset {section = {
    beforeskip = 3em
  }
}
% \ctexset {subsection = {
%     aftername = {、}
%   }
% }
% \ctexset {subsubsection = {
%     aftername = {}
%   }
% }

\usepackage{array}
\usepackage{tabularray}
\usepackage{amsmath}
\usepackage{amssymb}
\usepackage{amsthm}
\usepackage{calc}
\usepackage{caption}
\captionsetup[figure]{labelsep=none}

\usepackage{diagbox}
\usepackage{enumitem}
\usepackage{etoolbox}
\usepackage{float}
\usepackage{geometry}
\geometry{a4paper,left=2cm,right=2cm,top=2cm,bottom=1cm}
\usepackage{multicol}

\usepackage{hyperref}
\hypersetup{
  colorlinks=false, % 内部链接颜色
  pdfborder={0 0 0} % 移除链接框架
}

\usepackage{longdivision}
\usepackage{makecell}
\usepackage{multirow}
\usepackage{polynom}
\usepackage[figuresright]{rotating}
\usepackage{wrapfig}
\usepackage{tikz}
\usetikzlibrary{
  arrows.meta,
  calc,
  decorations.pathmorphing,
  intersections,
  patterns,
  patterns.meta,
  through,
}

\usepackage{xlop}
\usepackage{xparse}


% 绘制带圈的数字
\newcommand{\circled}[2][]{\tikz[baseline=(char.base)]
    {\node[shape = circle, draw, inner sep = 1pt]
    (char) {\phantom{\ifblank{#1}{#2}{#1}}};%
    \node at (char.center) {\makebox[0pt][c]{#2}};}}
\robustify{\circled}

% 罗马数字
\makeatletter
\newcommand{\rmnum}[1]{\romannumeral #1}
\newcommand{\Rmnum}[1]{\expandafter\@slowromancap\romannumeral #1@}
\makeatother

\newcommand{\moveup}{\setlength\abovedisplayskip{-1.5em}} % 当 “解:” 后直接写公式时,为了“解”与公式水平对齐所需要的偏移。
\newcommand{\practice}{\textbf{练\quad 习\ }}
\newcommand{\ulspace}[1][2em]{\underline{\hspace{#1}}}
\newcommand{\bs}[1]{\boldsymbol{#1}}

\setcounter{secnumdepth}{3}
\renewcommand{\thesubsection}{\thesection.\arabic{subsection}}
\renewcommand{\thesubsubsection}{\arabic{subsubsection}}
\renewcommand{\thefigure}{\thechapter.\arabic{figure}}
\ctexset {subsubsection = {
  aftername = {.\ }
  }
}

% 更改目录设置
\setcounter{tocdepth}{2}

\renewcommand{\thefootnote}{\circled{\arabic{footnote}}}
\counterwithout*{footnote}{chapter}

% 例题环境
\newcounter{cnteg}[chapter]
\newcommand{\eg}{
  \stepcounter{cnteg}
  \textbf{例\thechapter.\thecnteg\ }
}

% 解环境
\newcommand{\solve}{\textbf{【解】}}
\newcommand{\solved}{{\hfill 【解毕】}}

% 证明环境
\newcommand{\prove}{\textbf{【证明】}}
\newcommand{\proved}{{\hfill 【证毕】}}

% 习题环境
\newcommand{\exercise}{
  \vspace{1em}
  \newpage
  \begin{center}
    \bf\large{\labelexercise}
  \end{center}
}
\newcommand{\labelexercise}{ 
  习题\thesection
}

% 答案环境
\newcommand{\answer}{
  \vspace{1em}
  \newpage
  \setcounter{cntxiaoti}{0}
  \begin{center}
    \bf\large{\labelanswer}
  \end{center}
}
\newcommand{\labelanswer}{ 
  习题\thesection \quad 答案
}

% 小结环境
\newcommand{\summary}{
  \nonumsection[\labelsummary]{\huge \labelsummary}
}
\newcommand{\labelsummary}{小 \hspace{2em} 结}

\newcommand{\spacing}{\vspace{0.5em}} % 手工调整垂直间隔(测试发现 0.5em 比较合适。不支持参数,特殊情况直接使用 \vspace{} 命令。)

% 修改数学公式与上下文的距离
\makeatletter
\renewcommand\normalsize{%
    \abovedisplayskip 1\p@ \@plus1\p@ \@minus6\p@
    \belowdisplayskip \abovedisplayskip
}
\makeatother

% 绘制 弧 符号。代码来自
%    https://tex.stackexchange.com/questions/96680/
\makeatletter
\DeclareFontFamily{U}{tipa}{}
\DeclareFontShape{U}{tipa}{m}{n}{<->tipa10}{}
\newcommand{\myarc@char}{{\usefont{U}{tipa}{m}{n}\symbol{62}}}%

\newcommand{\myarc}[1]{\mathpalette\myarc@arc{#1}}

\newcommand{\myarc@arc}[2]{%
  \sbox0{$\m@th#1#2$}%
  \vbox{
    \hbox{\resizebox{\wd0}{\height}{\myarc@char}}
    \nointerlineskip
    \box0
  }%
}
\makeatother

% % ----------------------------------
% \newlength{\defaultParIndent}  % 页面缺省的 \parindent 长度。
% \setlength{\defaultParIndent}{\parindent}

% ----------------------------------
\theoremstyle{definition}
\newtheorem{theorem}{定理}[chapter]
\newtheorem{definition}{定义}[chapter]

% 小题和小小题
\newcounter{cntxiaoti}[subsubsection]      % 小题的计数器
\newcounter{cntxiaoxiaoti}[cntxiaoti]      % 小小题的计数器
\counterwithin*{cntxiaoxiaoti}{cnteg}

\newlength{\lenLabel}                     % 内部变量:用于计算题目前编号所占的长度
\newlength{\lenParent}                    % 内部变量:用于记录父题目编号所占的长度
\setlength{\lenParent}{0em}

\newenvironment{xiaotis}{ % “小题” 环境
  \NewDocumentCommand \xiaoti {s m} { % 小题的标题
    \IfBooleanTF {##1}
      {
        \setlength{\lenLabel}{\widthof{\labelxiaoti}}
        % \hangafter 1
        \setlength{\hangindent}{\parindent + \lenLabel}
        {\hspace{\lenLabel}##2}
      }
      {
        \stepcounter{cntxiaoti}
        \setlength{\lenLabel}{\widthof{\labelxiaoti}}
        % \hangafter 1
        % \setlength{\hangindent}{\parindent + \lenLabel}
        {\labelxiaoti ##2}
      }
  }
  \newcommand{\labelxiaoti}{\arabic{cntxiaoti}. }  % 1. 2. 3. ……
}{%
}

% 为命令 `\xiaoxiaoti' 增加一个可选参数,是为了实现:当“小题”没有文字时,第一个“小小题” 与“小题”同行。
% 对应 “习题 十四” 中的第 10 小题。
% 在此之前,是通过将 “小小题” 上移一行来实现同行显示。如: “习题 七” 中的第 10 小题。
% 但当“小题”本身处于页的最后一行时,上移不能生效,会导致“小题”显示在上一页的页末,而“小小题”显示在下一页的页首。
\newenvironment{xiaoxiaotis}{ % “小小题” 环境
  \setlength{\lenParent}{\lenParent + \lenLabel}
  \NewDocumentCommand{\xiaoxiaoti}{o m} { % 小小题的标题
    \stepcounter{cntxiaoxiaoti}
    \setlength{\lenLabel}{\widthof{\labelxiaoxiaoti}}
    \IfNoValueTF{##1}
      {\hangafter 1\setlength{\hangindent}{\parindent + \lenParent + \lenLabel + 0.5em}{\hspace{\lenParent}\labelxiaoxiaoti##2}}
      {\hangafter 1\setlength{\hangindent}{\parindent + \lenParent + \lenLabel + 0.5em}{\hspace{##1}\labelxiaoxiaoti##2}}
  }
  \newcommand{\labelxiaoxiaoti}{(\arabic{cntxiaoxiaoti})} % (1) (2) (3) ……
}{%
}

\newcommand{\twoInLine}  [3][10em] {\begin{tabular}[t]{*{2}{@{}p{#1}}} #2 & #3\end{tabular}}
\newcommand{\threeInLine}[4][10em] {\begin{tabular}[t]{*{3}{@{}p{#1}}} #2 & #3 & #4\end{tabular}}
\newcommand{\fourInLine} [5][10em] {\begin{tabular}[t]{*{4}{@{}p{#1}}} #2 & #3 & #4 & #5\end{tabular}}

\newcommand{\twoInLineXxt}  [3][10em] {\begin{tabular}[t]{*{2}{@{}p{#1}}} \xiaoxiaoti{#2} & \xiaoxiaoti{#3}\end{tabular}}
\newcommand{\threeInLineXxt}[4][10em] {\begin{tabular}[t]{*{3}{@{}p{#1}}} \xiaoxiaoti{#2} & \xiaoxiaoti{#3} & \xiaoxiaoti{#4}\end{tabular}}
\newcommand{\fourInLineXxt} [5][10em] {\begin{tabular}[t]{*{4}{@{}p{#1}}} \xiaoxiaoti{#2} & \xiaoxiaoti{#3} & \xiaoxiaoti{#4} & \xiaoxiaoti{#5}\end{tabular}}

\newcommand{\shangyihang}{\vspace{-1.5em}} % 上移一行

\newcommand{\degree}{^\circ}

\newcommand{\fourchoices}[5][10em]{\begin{tabular}[t]{*{4}{@{}p{#1}}} A.\ #2 &B.\ #3 &C.\ #4 &D.\ #5\end{tabular}}
\newcommand{\twochoices}[5][20em]{\begin{tabular}[t]{*{4}{@{}p{#1}}} A.\ #2 &B.\ #3 \\C.\ #4 &D.\ #5\end{tabular}}
\newcommand{\onechoices}[5][40em]{\begin{tabular}[t]{*{4}{@{}p{#1}}} A.\ #2 \\B.\ #3 \\C.\ #4 \\D.\ #5\end{tabular}}

% 修改itemize环境的行间距
\makeatletter
\def\@listi{\leftmargin\leftmargini
            \topsep 0pt
            \parsep 0pt
            \itemsep 0pt}
\makeatother

\newcommand{\R}{\mathbb{R}}
\renewcommand{\d}{\text{d}}
\renewcommand{\frac}[2]{\displaystyle\dfrac{#1}{#2}}


% 调整行间距
\renewcommand{\baselinestretch}{1.5}  % 1.5 倍行距

\makeatletter
\g@addto@macro\normalsize{%
  \setlength\abovedisplayskip{0.25em plus 0.5em minus 0.25em}%
  \setlength\belowdisplayskip{0.25em plus 0.5em minus 0.25em}%
  % \setlength\abovedisplayshortskip{0.5em plus 2pt minus 2pt}%
  % \setlength\belowdisplayshortskip{8pt plus 2pt minus 2pt}%
}
\makeatother
